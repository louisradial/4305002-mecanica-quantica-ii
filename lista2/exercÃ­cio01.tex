% vim: spl=pt
\begin{exercício}{Transformação entre bases de helicidade}{ex1}
   Consideremos um sistema de duas partículas com spin \(j_1\) e \(j_2\) no referencial de centro de momento. Seja a base de helicidade esférica \(\ket{jm \lambda_1 \lambda_2},\) onde \(j\) é o momento angular total, \(m\) é a projeção do momento angular total no eixo \(\vetor{e}_3\), e \(\lambda_1, \lambda_2\) são as helicidades das partículas, com normalização dada por
   \begin{equation*}
      \braket{j'm'\lambda_1'\lambda_2'}{jm\lambda_1 \lambda_2} = \delta_{jj'} \delta_{mm'} \delta_{\lambda_1 \lambda_1'} \delta_{\lambda_2 \lambda_2'}.
   \end{equation*}
   Seja a base de helicidade de onda plana \(\ket{\theta \phi \lambda_1 \lambda_2},\) onde \(\theta\) e \(\phi\) são os ângulos que definem a direção da partícula 1, com normalização
   \begin{equation*}
      \braket{\theta' \phi' \lambda_1' \lambda_2'}{\theta \phi \lambda_1 \lambda_2} = \delta(\Omega' - \Omega) \delta_{\lambda_1\lambda_1'} \delta_{\lambda_2 \lambda_2'},
   \end{equation*}
   onde \(\delta(\Omega' - \Omega)\) é o elemento de ângulo sólido para a partícula 1. Mostre que
   \begin{equation*}
      \ket{\theta\phi \lambda_1 \lambda_2} = \sum_{j, m} c_j \ket{jm \lambda_1 \lambda_2} D^j_{m \alpha}(\phi, \theta, -\phi),
   \end{equation*}
   onde \(\alpha = \lambda_1 - \lambda_2\) e determine \(c_j\).
\end{exercício}
\begin{proof}[Resolução]
   No centro de momento, temos \(\vetor{p}_1 = p \vetor{n} = - \vetor{p}_2,\) portanto 
   \begin{equation*}
      \vetor{J}\cdot\vetor{n} = \vetor{J}_1 \cdot \vetor{n} + \vetor{J}_2 \cdot \vetor{n} = \vetor{J}_1 \cdot \frac{\vetor{p}_1}{p_1} - \vetor{J}_2 \cdot \frac{\vetor{p}_2}{p_2} = h_1 - h_2,
   \end{equation*}
   e, assim, o autovalor de \(\vetor{J}\cdot\vetor{n}\) é \(\alpha = \lambda_1 - \lambda_2.\) Se \(\vetor{n}\) é descrito pelos ângulos \(\theta\) e \(\phi,\) isto é,
   \begin{equation*}
      \vetor{n} = \cos\phi\sin\theta\vetor{e}_x + \sin\phi\sin\theta \vetor{e}_y + \cos\theta \vetor{e}_z,
   \end{equation*}
   então
   \begin{equation*}
      \ket{\vetor{n} \lambda_1 \lambda_2} = D(\vetor{n}) \ket{\vetor{e}_z \lambda_1 \lambda_2},
   \end{equation*}
   com \(D(\vetor{n}) = e^{-i\phi J_z}e^{-i\theta J_y} e^{i\phi J_z}.\) Ainda, \(\ket{\vetor{e}_z \lambda_1 \lambda_2}\) é autovetor de \(J_z\) com autovalor \(\alpha\), então
   \begin{equation*}
      \ket{\vetor{e}_z \lambda_1 \lambda_2} = \sum_{j = \abs{\alpha}}^\infty \ket{j \alpha \lambda_1 \lambda_2} \braket{j \alpha \lambda_1 \lambda_2}{\vetor{e}_z \lambda_1 \lambda_2}
   \end{equation*}
   e obtemos
   \begin{align*}
      \ket{\vetor{n} \lambda_1 \lambda_2} = \sum_{j = \abs{\alpha}}^\infty D(\vetor{n}) \ket{j \alpha \lambda_1 \lambda_2} \braket{j \alpha \lambda_1 \lambda_2}{\vetor{e}_z \lambda_1 \lambda_2}
                                          =\sum_{j = \abs{\alpha}}^\infty \sum_{m = -j}^{j} D^{(j)}_{m \alpha}(\vetor{n}) \ket{j m \lambda_1 \lambda_2} \braket{j \alpha \lambda_1 \lambda_2}{\vetor{e}_z \lambda_1 \lambda_2},
   \end{align*}
   portanto \(c_j = \braket{j \alpha \lambda_1 \lambda_2}{\vetor{e}_z \lambda_1 \lambda_2}.\)

   Para determinar \(c_j,\) notemos que
   \begin{equation*}
      c_j D^{(j)}_{m \alpha}(\vetor{n}) = \braket{j m \lambda_1 \lambda_2}{\vetor{n} \lambda_1 \lambda_2}
   \end{equation*}
   portanto
   \begin{equation*}
      c_j c_{j'}^* D^{(j)}_{m \alpha}(\vetor{n}) D^{(j')}_{m' \alpha}(\vetor{n})^* = \braket{j m \lambda_1 \lambda_2}{\vetor{n} \lambda_1 \lambda_2} \braket{\vetor{n} \lambda_1 \lambda_2}{j' m' \lambda_1 \lambda_2}.
   \end{equation*}
   Da ortogonalidade na base de helicidade de onda plana, temos a relação de completeza
   \begin{equation*}
      \int \dli{\Omega'} \ketbra{\vetor{n'} \lambda_1 \lambda_2}{\vetor{n'} \lambda_1 \lambda_2} = \unity
   \end{equation*}
   no subespaço de helicidade \(\alpha = \lambda_1 - \lambda_2\),
   portanto integrando a relação anterior obtemos
   \begin{equation*}
      c_j c_{j'}^* \frac{4\pi}{2j + 1} \delta_{jj'} \delta_{mm'} = \braket{jm \lambda_1 \lambda_2}{j' m' \lambda_1 \lambda_2} = \delta_{jj'} \delta_{mm'},
   \end{equation*}
   isto é, 
   \begin{equation*}
      \abs{c_j}^2 = \frac{2j + 1}{4\pi}.
   \end{equation*}
   Com isso, tomamos \(c_j\) real e obtemos
   \begin{equation*}
      \ket{\vetor{n} \lambda_1 \lambda_2} = \sum_{j = \abs{\alpha}}^{\infty}{\sum_{m = -j}^{j}{\sqrt{\frac{2j + 1}{4\pi}} D^{(j)}_{m \alpha}(\vetor{n}) \ket{j m \lambda_1 \lambda_2}}}
   \end{equation*}
   como a transformação entre as bases.
\end{proof}

% vim: spl=pt
\begin{exercício}{Medida do momento magnético de \(\Lambda^0\)}{ex6}
   A partícula \(\Lambda^0\) tem carga nula, massa \(M \simeq \SI{1115}{\mega\eV/c^2},\) spin \(\frac12\) e tempo de vida \(\tau \simeq \SI{2.5e-10}{\second}.\) Um de seus modos de decaimento é \(\Lambda^0 \to p + \pi^-,\) onde o próton tem spin \(\frac12\) e o píon, spin nulo.
   \begin{enumerate}[label=(\alph*)]
      \item No referencial de repouso de \(\Lambda^0,\) podemos supor que o próton é emitido com momento \(\vetor{p} = p \vetor{e}_z,\) onde \(\vetor{e}_z\) é o eixo escolhido para a quantização do momento angular. Seja \(m\) a projeção do spin de \(\Lambda^0\) nesse eixo e \(m'\) a do próton. Por que devemos ter \(m' = m\)? Considere as probabilidades de transição \(a\) e \(b\) 
         \begin{equation*}
            a: \Lambda^0\left(m = \frac12\right) \to p \left(m' = \frac12; \vetor{p} \parallel \vetor{e}_z\right)
            \quad\text{e}\quad
            b: \Lambda^0\left(m = -\frac12\right) \to p \left(m' = -\frac12; \vetor{p} \parallel \vetor{e}_z\right).
         \end{equation*}
         Mostre que \(\abs{a} = \abs{b}\) se paridade é conservada no decaimento. \todo[Considere uma reflexão em relação ao plano xz.]
      \item O próton é agora emitido com momento \(\vetor{p} = p\vetor{n}\) na direção \(\vetor{n}\) contida no plano \(xz,\) com ângulo \(\theta\) em relação ao eixo \(\vetor{e}_z\). Considere \(m'\) a projeção do spin do próton nessa direção e \(a_{m'm}(\theta)\) a amplitude
         \begin{equation*}
            a_{m'm}(\theta) : \Lambda^0\left(m = \frac12\right) \to p \left(m'; \vetor{p} \parallel \vetor{n}\right).
         \end{equation*}
         Escreva \(a_{++}(\theta)\) e \(a_{-+}(\theta)\) em função de \(a, b,\) e \(\theta\).
      \item Suponha que \(\Lambda^0\) seja produzido no estado \(m = \frac12\). Mostre que a distribuição angular do próton é da forma \(\omega(\theta) = \omega_0 (1 + \alpha \cos\theta)\) e determina \(\alpha\) como função de \(a\) e de \(b.\) O experimento mostra que \(\alpha \simeq -0.645 \pm 0.016.\) O que podemos concluir sobre a conservação de paridade nesse decaimento?
      \item \(\Lambda^0\) é produzido por um feixe de \(\pi^-\) colidindo com um próton em repouso,
         \(\pi^- + p \to \Lambda^0 + K^0.\) Devido à conservação de momento \(\vetor{p}_\pi, \vetor{p}_\Lambda,\) e \(\vetor{p}_K\) são coplanares. Tomamos \(\vetor{e}_z\) como o eixo perpendicular a esse plano,
         \begin{equation*}
            \vetor{e}_z = \frac{\vetor{p}_\pi \times \vetor{p}_\Lambda}{\norm{\vetor{p}_\pi \times \vetor{p}_\Lambda}},
         \end{equation*}
         e tomamos \(\vetor{p}_\Lambda = p_\Lambda \vetor{e}_y.\) Sabendo que paridade é conservada na reação de produção e que os prótons não são polarizados, mostre que se \(\vetor{S}\) é o operador de spin de \(\Lambda^0,\) então \(\mean{S_x} = \mean{S_y} = 0.\)
      \item Vamos supor que \(\mean{S_z} = \frac12\) e que todos \(\Lambda^0\) têm o mesmo tempo de vida e que decaem no mesmo ponto. O sistema se encontra em um campo magnético \(\vetor{B} = B \vetor{e}_y\) constante e uniforme. \(\Lambda^0\) tem um momento magnético \(\vetor{\mu} = \gamma \vetor{S}.\) Descreva qualitativamente o movimento de spin. Determine a sua orientação no instante de decaimento como função de \(\tau,\) \(B\) e \(\gamma\). Mostre que a distribuição angular do próton emitido no decaimento é \(\omega(\theta, \phi) = \omega_0 (1 + \alpha \cos\Theta)\), com \(\cos\Theta = \cos \lambda \cos\theta + \sin \lambda \sin \theta \cos \phi,\) onde os ângulos \(\theta\) e \(\phi\) descrevem a direção do momento do próton. Qual é o valor de \(\lambda?\) Deduza que a determinação de \(\omega(\theta,\phi)\) permite a medida do fator giromagnético \(\gamma\). Desprezamos o efeito do campo magnético na trajetória do próton assim como as transformações dos ângulos devido ao movimento de \(\Lambda^0.\)
   \end{enumerate}
\end{exercício}
\begin{proof}[Resolução]
    
\end{proof}

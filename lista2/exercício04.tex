% vim: spl=pt
\begin{exercício}{Distribuição angular de decaimento II}{ex4}
   Uma partícula de spin 1 é polarizada de forma que seu spin tem direção \(\vetor{e}_z\). Ela decai com taxa total \(\Gamma\) para \(\pi^+\pi^-.\) Qual é a distribuição angular \(\diff{\Gamma}{\Omega}\) do \(\pi^+,\) dado que \(\pi^\pm\) tem spin nulo.
\end{exercício}
\begin{proof}[Resolução]
   Como os píons têm spin nulo, o estado final tem helicidade nula e momento angular \(\ell = 1.\) Assim, a distribuição angular de decaimento é
   \begin{equation*}
      \diff{\Gamma}{\Omega} \propto \abs{D^{(1)}_{10}(\vetor{n})}^2 = \frac12 \sin^2\theta,
   \end{equation*}
   e, portanto,
   \begin{equation*}
      \frac1\Gamma \diff{\Gamma}{\Omega} = \frac{3}{8\pi}\sin^2\theta
   \end{equation*}
   em termos da taxa total \(\Gamma.\)
\end{proof}

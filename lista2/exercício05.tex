% vim: spl=pt
\begin{exercício}{Distribuição angular de decaimento III}{ex5}
   Considere uma partícula de spin 1 polarizada com momento angular ao longo do eixo \(\pm \vetor{e}_z\) com probabilidades iguais. Suponha que ela decaia para duas partículas de spin \(\frac12.\)
   \begin{enumerate}[label=(\alph*)]
       \item Suponha que o decaimento ocorra com momento angular orbital nulo. Qual é a distribuição angular dos produtos de decaimento, no referencial da partícula que decai.
       \item Se esse processo é um decaimento eletromagnético para \(e^+ e^-\), e a massa da partícula que decai é muito maior que a massa do elétron, a situação é alterada, segundo a QED. Nesse caso, os estados finais de spin serão orientados de tal forma que \(m = \pm 1\) ao longo do eixo de decaimento, onde \(m\) é a projeção do momento angular total. Qual é a distribuição angular dos produtos de decaimento nesse caso?
   \end{enumerate}
\end{exercício}
\begin{proof}[Resolução]
   Para um estado inicial com \(j = 1\) e \(M = \pm 1,\) devemos ter
   \begin{equation*}
      A(i \to f) \propto \frac12 \sum_{M \in \set{-1,1}} \sum_{\lambda_1 \in \set{-\frac12,\frac12}} \sum_{\lambda_2 \in \set{-\frac12,\frac12}} C_{\lambda_1 \lambda_2} D_{M \Lambda}^{(1)}(\vetor{n})
   \end{equation*}
   para a amplitude do decaimento em duas partículas de spin \(\frac12\). Assim,
   \begin{align*}
      \diff{\Gamma}{\Omega} &\propto \frac14 \sum_{M = \pm1} \sum_{\lambda_1} \sum_{\lambda_2} \abs{C_{\lambda_1 \lambda_2}}^2 \abs{D^{(1)}_{M \Lambda}(\vetor{n})}^2\\
                            &= \frac14 \sum_{M = \pm1}\left[\left(\abs{C_{++}}^2 + \abs{C_{--}}^2\right)\abs{D^{(1)}_{M0}(\vetor{n})}^2 + \abs{C_{+-}}^2\abs{D^{(1)}_{M1}(\vetor{n})}^2 + \abs{C_{-+}}^2 \abs{D^{(1)}_{M{-1}}(\vetor{n})}^2\right]\\
                            &= \frac14 \left\{\left(\abs{C_{++}}^2 + \abs{C_{--}}^2\right)\sin^2\theta +\left(\abs{C_{+-}}^2 + \abs{C_{-+}}^2\right)\left[\left(\frac{1 + \cos\theta}{2}\right)^2 + \left(\frac{1 - \cos\theta}{2}\right)^2\right]\right\}\\
                            &= \frac14 \left\{\left(\abs{C_{++}}^2 + \abs{C_{--}}^2\right)\sin^2\theta + \left(\abs{C_{+-}}^2 + \abs{C_{-+}}^2\right)\frac{1 + \cos^2\theta}{2}\right\},
   \end{align*}
   portanto em termos da taxa total,
   \begin{equation*}
      \frac{1}{\Gamma}\diff{\Gamma}{\Omega} = \frac{3}{16\pi}\frac{2\left(\abs{C_{++}}^2 + \abs{C_{--}}^2\right)\sin^2\theta + \left(\abs{C_{+-}}^2 + \abs{C_{-+}}^2\right)\left(1 + \cos^2\theta\right)}{\abs{C_{++}}^2 + \abs{C_{+-}}^2 + \abs{C_{-+}}^2 + \abs{C_{--}}^2}.
   \end{equation*}

   No caso em que o momento angular orbital é nulo, devemos ter apenas os casos em que \(M = \Lambda,\) já que esses estados correspondem aos estados de momento angular \(\ket{11}\) e \(\ket{1{-1}}.\)
   \begin{align*}
      \diff{\Gamma}{\Omega} &\propto \frac14 \left[\abs{C_{+-}}^2 \abs{D^{(1)}_{11}(\vetor{n})}^2 + \abs{C_{-+}}^2 \abs{D^{(1)}_{-1{-1}}(\vetor{n})}^2\right]\\
                            &= \frac14 \left[\abs{C_{+-}}^2 + \abs{C_{-+}}^2\right] \frac{1 + 2\cos\theta + \cos^2\theta}{4}
   \end{align*}
   isto é,
   \begin{equation*}
      \frac1\Gamma\diff{\Gamma}{\Omega} = \frac{3}{\pi} (1 + 2\cos\theta + \cos^2\theta).
   \end{equation*}

   No caso em que os estados finais de spin são tais que \(m = \pm 1,\) devemos ter o caso anterior além do caso de singleto de spin, em que a helicidades das partículas são iguais, logo
   \begin{align*}
      \diff{\Gamma}{\Omega} &\propto \frac14 \left[\abs{C_{+-}}^2 \abs{D^{(1)}_{11}(\vetor{n})}^2 + \abs{C_{-+}}^2 \abs{D^{(1)}_{-1{-1}}(\vetor{n})}^2 + \left(\abs{C_{++}}^2 + \abs{C_{--}}^2\right)\left(\abs{D^{(1)}_{10}(\vetor{n})}^2 + \abs{D^{(1)}_{{-1}0}(\vetor{n})}^2\right)\right]\\
                            &= \frac14 \left[\left(\abs{C_{+-}}^2 + \abs{C_{-+}}^2\right) \frac{1 + 2\cos\theta + \cos^2\theta}{4} + \left(\abs{C_{++}}^2 + \abs{C_{--}}^2\right)\sin^2\theta\right]
   \end{align*}
\end{proof}

% vim: spl=pt
\begin{exercício}{Distribuição angular de decaimento III}{ex5}
   Considere uma partícula de spin 1 polarizada com momento angular ao longo do eixo \(\pm \vetor{e}_z\) com probabilidades iguais. Suponha que ela decaia para duas partículas de spin \(\frac12.\)
   \begin{enumerate}[label=(\alph*)]
       \item Suponha que o decaimento ocorra com momento angular orbital nulo. Qual é a distribuição angular dos produtos de decaimento, no referencial da partícula que decai.
       \item Se esse processo é um decaimento eletromagnético para \(e^+ e^-\), e a massa da partícula que decai é muito maior que a massa do elétron, a situação é alterada, segundo a QED. Nesse caso, os estados finais de spin serão orientados de tal forma que \(m = \pm 1\) ao longo do eixo de decaimento, onde \(m\) é a projeção do momento angular total. Qual é a distribuição angular dos produtos de decaimento nesse caso?
   \end{enumerate}
\end{exercício}
\begin{proof}[Resolução]
    
\end{proof}

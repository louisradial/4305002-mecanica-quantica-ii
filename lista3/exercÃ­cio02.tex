% vim: spl=pt
\begin{exercício}{Potencial de Yukawa}{ex2}
   Considere o espalhamento pelo potencial de Yukawa
   \begin{equation*}
      V(r) = \frac1r\exp\left(-\frac{r}{a}\right).
   \end{equation*}
   \begin{enumerate}[label=(\alph*)]
       \item Determine a amplitude de espalhamento e a seção de choque total usando a aproximação de Born.
       \item Discuta a validade da aproximação de Born para o potencial de Yukawa impondo 
          \begin{equation*}
             \frac{2m}{\hbar^2}\abs*{\int\dln3r \frac{e^{ikr}}{4\pi r} V(\vetor{r}) e^{ikz}} \ll 1.
          \end{equation*}
       \item Mostre que a seção de choque total é menor que que a seção transversal geométrica \(4\pi a^2\) quando a aproximação de Born é válida independente de momento.
   \end{enumerate}
\end{exercício}
\begin{proof}[Resolução]
    
\end{proof}

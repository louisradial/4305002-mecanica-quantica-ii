% vim: spl=pt
\begin{exercício}{Espalhamento de Rutherford e o tamanho nuclear}{ex1}
   Calcule a amplitude e a seção de choque diferencial de elétrons com massa \(m\) e momento \(\vetor{p} = \hbar \vetor{k}\) espalhados por um núcleo na aproximação de Born. Assuma que a carga \(Ze\) do núcleo é esfericamente distribuída e que tem densidade \(\rho(r)\). Mostre que a amplitude de espalhamento pode ser escrita como
   \begin{equation*}
       f(\theta) = f_0(\theta) F(k),
   \end{equation*}
   onde \(f_0(\theta)\) é a amplitude de espalhamento de Rutherford,
   \begin{equation*}
      f_0(\theta) = \frac{Ze^2}{4E\sin^2\frac{\theta}{2}}
   \end{equation*}
   e \(F(k)\) é o fator de forma
   \begin{equation*}
      F(k) = \frac{4\pi}{Ze} \int_0^\infty \dli{r} r^2 \rho(r) \frac{\sin kr}{kr}.
   \end{equation*}
   Assuma agora que a densidade do núcleo é aproximadamente constante e :
\end{exercício}
\begin{proof}[Resolução]
    
\end{proof}

% vim: spl=pt
\begin{exercício}{Poço de potencial esférico}{ex3}
   Considere um poço de potencial esférico,
   \begin{equation*}
       V(r) = \begin{cases}
          -V_0,&r < a,\\
          0, & r > a.
       \end{cases}
   \end{equation*}
   Considerando a onda parcial \(\ell = 0,\)
   \begin{enumerate}[label=(\alph*)]
       \item determine a condição para os estados ligados e faça um gráfico das energias de estados ligados em função de \(V_0\);
       \item calcule a defasagem \(\delta_0\) e a matriz \(S\) \(e^{2i \delta_0}\);
       \item mostre que o polo da matriz \(S\) no semiplano superior de \(k\) corresponde aos estados ligados; e
       \item faça um gráfico da seção de choque \(\sigma_0\) como função de \(ka\) para diferentes valores de \(V_0\) para mostrar o comportamento longe, logo abaixo, exatamente e logo acima do limiar de estado ligado, comparando com a seção transversal geométrica \(4\pi a^2.\)
   \end{enumerate}
   Para ondas parciais com \(\ell = 1,\)
   \begin{enumerate}[label=(\alph*)]
       \item considere um momento pequeno \(ka = 0.1\) e faça um gráfico de \(\sin^2\delta_1\) como função do potencial \(V_0,\) e identifique \(V_0\) nos picos agudos;
       \item faça um gráfico de \(\sin^2\delta_1\) como função de \(k\) para um \(V_0\) fixo determinado no item anterior; e
       \item faça um gráfico da função de onda radial \(R_1(r)\) para a combinação de \(k\) e \(V_0\) que corresponde ao pico da seção de choque.
   \end{enumerate}
\end{exercício}
\begin{proof}[Resolução]
    
\end{proof}

% vim: spl=pt
\begin{exercício}{Expansão em ondas parciais com interação spin-órbita\cite{gottfried}}{ex5}
   Considere o espalhamento elástico devido à interação spin-órbita descrita pelo potencial
   \begin{equation*}
      V = V_0(r) + \vetor{\sigma} \cdot \vetor{L} V_1(r).
   \end{equation*}
   \begin{enumerate}[label=(\alph*)]
       \item Mostre que a solução da equação de evolução temporal pode ser escrita como
          \begin{equation*}
             \ket{\psi} = \sum_{\ell = 0}^{\infty} \sqrt{4\pi (2\ell + 1)} i^\ell \left[c_\ell^+ R_{\ell}^+(r) \Lambda_{\ell}^+ + c_{\ell}^- R_{\ell}^-(r) \Lambda_\ell^-\right]Y_{\ell 0}(\theta) \ket{\nu_i},
          \end{equation*}
          onde \(\Lambda_{\ell}^{\pm}\) são os projetores ortogonais nos subespaços \(j = \ell \pm \frac12\), dados por
          \begin{equation*}
             \Lambda_{\ell}^+ = \frac{\ell + 1 + \vetor{\sigma}\cdot\vetor{L}}{2\ell + 1}\quad\text{e}\quad
             \Lambda_{\ell}^- = \frac{\ell - \vetor{\sigma}\cdot\vetor{L}}{2\ell + 1},
          \end{equation*}
          \(c_{\ell}^\pm\) são constantes determinadas pelas condições de contorno, e as funções radiais \(R_{\ell}^\pm\) são soluções de
          \begin{equation*}
             \left[\frac{1}{r^2} \diff{}{r}r^2 \diff{}{r} - \frac{\ell (\ell + 1)}{r^2} + k^2 - 2m V_{\ell}^{\pm}(r)\right]R_{\ell}^{\pm}(r) = 0,
          \end{equation*}
          com
          \begin{equation*}
             V_{\ell}^{+} = V_0(r) + \ell V_1(r)\quad\text{e}\quad V_{\ell}^{-} = V_0(r) - (\ell + 1)V_1(r).
          \end{equation*}
       \item Seja \(\delta_{\ell}^{\pm}\) as defasagens nos estados de momento angular total \(j = \ell \pm \frac12,\) isto é,
          \begin{equation*}
             R_{\ell}^\pm \sim \frac{e^{i \delta_{\ell}^\pm}}{kr}\sin\left(kr - \ell \frac\pi2 + \delta_\ell^\pm\right)
          \end{equation*}
          no limite assintótico \(r \to \infty.\) Mostre que as funções \(g\) e \(h\) que aparecem na matriz de espalhamento \(M\) são
          \begin{equation*}
             g(k,\theta) = \frac1k \sum_{\ell = 0}^\infty \left(\frac{4\pi}{2\ell + 1}\right)^{\frac12} \left[(\ell + 1) a_\ell^+ + \ell a_\ell^-\right]Y_{\ell 0}(\theta)
          \end{equation*}
          e
          \begin{equation*}
             h(k,\theta) = \frac1k \sum_{\ell = 0}^\infty \left(\frac{4\pi}{2\ell + 1}\right)^{\frac12} \left[a_\ell^+ - a_\ell^-\right]i\sin\theta\diff*{Y_{\ell 0}(\theta)}{\cos\theta},
          \end{equation*}
          onde \(a_{\ell}^\pm = e^{i \delta_{\ell}^\pm} \sin\delta_{\ell}^{\pm}.\)
       \item Mostre que a seção de choque total é
          \begin{equation*}
             \sigma = \frac{4\pi}{k^2} \sum_{\ell} \left[(\ell + 1)\sin\delta^+_\ell + \ell \sin^2 \delta_\ell^-\right].
          \end{equation*}
   \end{enumerate}
\end{exercício}
\begin{proof}[Resolução]
   Como \(\vetor{\sigma} \cdot \vetor{L} = \vetor{J}^2 - \vetor{L}^2 - \vetor{S}^2\) temos
   \begin{equation*}
      \vetor{\sigma} \cdot \vetor{L} \ket{j \ell m} = \frac12 \left[\pm(2\ell + 1) - 1\right] \ket{j \ell m}
      \implies \Lambda_{\ell}^{\pm}\ket{j \ell m} = \delta_{j, \ell \pm \frac12} \ket{j \ell m}.
   \end{equation*}
   portanto
   \begin{equation*}
      \vetor{\sigma} \cdot \vetor{L} \ket{j \ell m} = \frac12\left[(2\ell + 1) \left(\Lambda_\ell^+ - \Lambda_\ell^-\right) - 1\right]\ket{j \ell m} = \left[\ell \Lambda_\ell^+ - (\ell + 1)\Lambda_\ell^-\right]\ket{j \ell m}.
   \end{equation*}
   Assim, podemos escrever
   \begin{equation*}
      V\ket{j\ell m} = \left[V^+_\ell \Lambda^+_\ell + V^-_{\ell} \Lambda^{-}_\ell\right]\ket{j\ell m},
   \end{equation*}
   onde 
   \begin{equation*}
      V^+_\ell = V_0(r) + \ell V_1(r)
      \quad\text{e}\quad
      V^-_\ell = V_0(r) - (\ell+1) V_1(r).
   \end{equation*}

   Pela invariância de rotação em torno de \(\vetor{k},\) temos
   \begin{equation*}
      \braket{\vetor{r}}{\psi_{\vetor{k}}} = \sum_{\ell = 0}^\infty (2\ell + 1)i^\ell c_\ell R_\ell(r) P_\ell\left(\frac{\vetor{r}}{r}\cdot \frac{\vetor{k}}{k}\right) \ket{\nu_i} = \sum_{\ell = 0}^\infty \sqrt{4\pi (2\ell + 1)} i^\ell c_\ell R_{\ell}(r) Y_{\ell 0}(\theta) \ket{\nu_i},
   \end{equation*}
   onde \(\theta\) é o ângulo entre \(\vetor{r}\) e \(\vetor{k}\). Decompondo \(\ket{\nu_i} = (\Lambda_\ell^{+} + \Lambda_\ell^{-})\ket{\nu_i},\) temos
   \begin{equation*}
      \braket{\vetor{r}}{\psi_{\vetor{k}}} = \sum_{\ell = 0}^\infty \sqrt{4\pi (2\ell + 1)} i^{\ell} \left[c_\ell^+R_\ell^{+}(r) \Lambda_{\ell}^+ + c_\ell^{-} R_{\ell}^{-}(r) \Lambda_{\ell}^-\right]Y_{\ell 0}(\theta) \ket{\nu_i}.
   \end{equation*}
   Exigindo que \(H \ket{\psi_{\vetor{k}}} = \frac{k^2}{2m}\ket{\psi_{\vetor{k}}}\) devemos ter que
   \begin{equation*}
      \left[\frac{1}{r^2} \diff{}{r}r^2 \diff{}{r} - \frac{\ell (\ell + 1)}{r^2} + k^2 - 2m V_{\ell}^{\pm}(r)\right]R_{\ell}^{\pm}(r) = 0,
   \end{equation*}
   como facilmente se verifica.

   Para determinar \(M\) escrevemos a forma assintótica do estado de espalhamento como
   \begin{equation*}
      \Psi \sim e^{i \vetor{k}_i \cdot \vetor{r}} \ket{\nu_i} + \frac{e^{ikr}}{r} M(\vetor{k}_f, \vetor{k_i}) \ket{\nu_i}.
   \end{equation*}
   Assim, utilizando a expansão
   \begin{equation*}
      e^{i \vetor{k} \cdot \vetor{r}} = \sum_{\ell = 0}^\infty \sqrt{4\pi(2\ell + 1)} i^\ell j_\ell(kr) Y_{\ell0}(\theta) \mathop{\sim}^{r\to\infty} \sum_{\ell = 0}^\infty \sqrt{4\pi(2\ell + 1)} i^\ell \frac{e^{i(kr - \frac12 \ell \pi)} - e^{-i(kr - \frac12 \ell \pi)}}{2ikr} Y_{\ell0}(\theta)
   \end{equation*}
   temos da expressão da função de onda do estado assintótico
   \begin{align*}
      \psi &= \sum_{\ell=0}^\infty \sqrt{4\pi(2\ell + 1)} i^\ell \left(\sum_{s \in \set{-,+}} c^s_{\ell} R_\ell^s(r) \Lambda_\ell^s \right) Y_{\ell 0}(\theta) \ket{\nu_i}\\
           &\mathop{\sim}^{r\to\infty} \sum_{\ell = 0}^\infty \sqrt{4\pi(2\ell + 1)} i^{\ell} \left[\sum_{s \in\set{-,+}}c_\ell^s \frac{e^{i \delta_\ell^s}}{kr}\left(\frac{e^{i(kr - \ell \frac\pi2 + \delta_\ell^s)} - e^{-i(kr - \ell \frac\pi2 + \delta_{\ell}^s)}}{2i}\right)\Lambda^s_\ell\right] Y_{\ell 0}(\theta) \ket{\nu_i}\\
           &= \sum_{\ell = 0}^\infty \sqrt{4\pi(2\ell + 1)} i^{\ell} \left[\sum_{s \in\set{-,+}}c_\ell^s \left(\frac{e^{2i \delta_{\ell}^s}e^{i(kr - \ell \frac\pi2)} - e^{-i(kr - \ell \frac\pi2)}}{2ikr}\right)\Lambda^s_\ell\right] Y_{\ell 0}(\theta) \ket{\nu_i}\\
           &= \sum_{\ell = 0}^\infty \sqrt{4\pi(2\ell + 1)} i^{\ell} \left[\sum_{s \in\set{-,+}}c_\ell^s \left(\frac{e^{i(kr - \ell \frac\pi2)} - e^{-i(kr - \ell \frac\pi2)}}{2ikr}\right)\Lambda^s_\ell\right] Y_{\ell 0}(\theta) \ket{\nu_i} + \\
           &{}\phantom{= + } + 
           \sum_{\ell = 0}^\infty \sqrt{4\pi(2\ell + 1)} i^{\ell} \left[\sum_{s \in\set{-,+}}c_\ell^s \frac{\left(e^{2i \delta_{\ell}^s} - 1\right)e^{i(kr - \ell \frac\pi2)}}{2ikr}\Lambda^s_\ell\right] Y_{\ell 0}(\theta) \ket{\nu_i}\\
           &\mathop{\sim}^{r\to\infty}\sum_{\ell = 0}^\infty \sqrt{4\pi(2\ell + 1)} i^{\ell} j_{\ell}(kr)\left[\sum_{s \in\set{-,+}}c_\ell^s \Lambda^s_\ell\right] Y_{\ell 0}(\theta) \ket{\nu_i} +\\
           &{}\phantom{\mathop{\sim}^{r\to\infty} + } + \sum_{\ell = 0}^\infty \sqrt{4\pi(2\ell + 1)} i^{\ell} \left[\sum_{s \in\set{-,+}}c_\ell^s \frac{\left(e^{2i \delta_{\ell}^s} - 1\right)e^{i(kr - \ell \frac\pi2)}}{2ikr}\Lambda^s_\ell\right] Y_{\ell 0}(\theta) \ket{\nu_i}
   \end{align*}
   que \(c_\ell^{\pm} = 1\) e, portanto,
   \begin{align*}
      \psi &= e^{i \vetor{k}\cdot\vetor{r}} \ket{\nu_i} + \frac{e^{ikr}}{kr}\sum_{\ell = 0}^\infty \sqrt{4\pi(2\ell + 1)} \left[\sum_{s \in \set{-,+}} e^{i \delta_{\ell}^s}\sin(\delta_{\ell}^s) \Lambda_{\ell}^{s}\right] Y_{\ell 0}(\theta)\ket{\nu_i}\\
           &= e^{i \vetor{k}\cdot\vetor{r}} \ket{\nu_i} + \frac{e^{ikr}}{kr}\sum_{\ell = 0}^\infty \sqrt{4\pi(2\ell + 1)}  \left(a_{\ell}^+ \Lambda_\ell^+ + a_{\ell}^- \Lambda_\ell^-\right)Y_{\ell 0}(\theta)\ket{\nu_i},
   \end{align*}
   onde definimos \(a_{\ell}^s = e^{i \delta_{\ell}^s} \sin(\delta_{\ell}^{s}).\) Notemos que
   \begin{equation*}
      a_\ell^+ \Lambda_{\ell}^+ + a_{\ell}^- \Lambda_{\ell}^- = \frac{(\ell + 1)a_{\ell}^+ + \ell a_{\ell}^{-}}{2\ell + 1} + \frac{a_{\ell}^+ - a_{\ell}^{-}}{2\ell + 1} \vetor{\sigma}\cdot\vetor{L},
   \end{equation*}
   e notemos que, definindo \(\theta\) como o ângulo polar e \(\phi\) o ângulo azimutal a partir dos eixos \(\frac{\vetor{k}_i}{k}\) e \(\frac{\vetor{k}_f}{k}\) respectivamente, temos
   \begin{align*}
      \vetor{\sigma}\cdot \vetor{L} Y_{\ell 0}(\theta) \ket{\nu_i} 
      &= \sigma_0\ket{\nu_i} L_0 Y_{\ell 0}(\theta) - \sigma_{+1}\ket{\nu_i} L_{-1} Y_{\ell 0}(\theta) - \sigma_{-1}\ket{\nu_i} L_{+1}Y_{\ell 0}(\theta)\\
      &= - \sigma_{+1}\ket{\nu_i} \left[- \frac{1}{\sqrt{2}}\diff{Y_{\ell 0}(\theta)}{\theta}\right] - \sigma_{-1} \ket{\nu_i} \left[- \frac{1}{\sqrt{2}}\diff{Y_{\ell 0}(\theta)}{\theta}\right]\\
      &= \frac{\sigma_{+1} + \sigma_{-1}}{\sqrt{2}} \ket{\nu_i} \diff{Y_{\ell 0}(\theta)}{(\cos\theta)} \diff{\cos\theta}{\theta}\\
      &= i  \sin\theta\diff{Y_{\ell 0}(\theta)}{(\cos\theta)}\sigma_2 \ket{\nu_i},
   \end{align*}
   portanto com os eixos definidos \(\sigma_2\) corresponde a \(\vetor{\sigma} \cdot \vetor{n},\) onde \(\vetor{n} = \frac{\vetor{k}_i \times \vetor{k}_f}{\norm{\vetor{k}_i\times\vetor{k}_f}}\). Assim, obtemos
   \begin{equation*}
      \psi = e^{i\vetor{k}\cdot\vetor{r}}\ket{\nu_i} + \frac{e^{ikr}}{kr} \sum_{\ell = 0}^\infty \sqrt{\frac{4\pi}{2\ell + 1}} \left\{\left[(\ell + 1)a_{\ell}^+ + \ell a_{\ell}^-\right]Y_{\ell 0}(\theta) + i(a_{\ell}^+ - a_{\ell}^-)\sin\theta \diff{Y_{\ell0}(\theta)}{(\cos\theta)} \vetor{n}\cdot\vetor{\sigma}\right\}\ket{\nu_i},
   \end{equation*}
   logo a matriz de espalhamento \(M\) é dada por
   \begin{equation*}
      M = g(k,\theta) + h(k,\theta)\vetor{n}\cdot\vetor{\sigma},
   \end{equation*}
   com
   \begin{equation*}
      g(k,\theta) = \frac{1}{k}\sum_{\ell = 0}^\infty \sqrt{\frac{4\pi}{2\ell + 1}} \left[(\ell + 1)a_{\ell}^+ + \ell a_{\ell}^-\right]Y_{\ell 0}(\theta)
      \;\;\text{e}\;\;
      h(k,\theta) = \frac{i}{k}\sum_{\ell = 0}^\infty \sqrt{\frac{4\pi}{2\ell + 1}} (a_{\ell}^+ - a_{\ell}^-)\sin\theta \diff{Y_{\ell0}(\theta)}{(\cos\theta)}.
   \end{equation*}
\end{proof}

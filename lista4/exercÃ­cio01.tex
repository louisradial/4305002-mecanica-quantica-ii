% vim: spl=pt
\begin{exercício}{Espalhamento férmion-férmion}{ex1}
   A forma mais geral de uma matriz densidade para o sistema nêutron-próton é
   \begin{equation*}
      \rho = \frac14 \left(\unity + \vetor{\sigma}_n \cdot \vetor{P}_n + \vetor{\sigma}_p \cdot \vetor{P}_p + \sum_{ij}C_{ij}{\sigma_{n,i}\sigma_{p,j}}\right)
   \end{equation*}
   onde \(\vetor{P}_n,\vetor{P}_p\) são, respectivamente, as polarizações de spin do próton e do nêutron.
   \begin{enumerate}[label=(\alph*)]
      \item Mostre que mesmo que o estado inicial seja aleatório, \(\rho_i = \frac14\), há correlação entre os estados de spin após o espalhamento. Determine os coeficientes de correlação \(C_{ij}\) em termos da matriz de espalhamento \(\mathcal{M}(\vetor{k}_f,\vetor{k}_i)\) nesse caso.
      \item Novamente nesse caso, calcule a seção de choque diferencial elástica considerando a interação spin-spin,
         \begin{equation*}
            V = V_0(r) + \vetor{\sigma}_1 \cdot \vetor{\sigma}_2 V_1(r),
         \end{equation*}
         em termos das amplitudes \(f_{s}(\theta, k)\) e \(f_t(\theta,k).\)
      \item Determine os coeficientes \(C_{ij}\) em termos de \(f_s(\theta,k)\) e \(f_t(\theta,k).\)
      \item Considere agora o espalhamento próton-próton com o mesmo potencial de interação. Se o estado inicial é não polarizado, mostre que os coeficientes de correlação são \(C_{ij} = \delta_{ij} C,\)
         \begin{equation*}
            C = \frac{\abs{f_t(\theta) - f_t(\pi - \theta)}^2 - \abs{f_s(\theta) + f_s(\pi - \theta)}^2}{3 \abs{f_t(\theta) - f_t(\pi - \theta)}^2 + \abs{f_s(\theta) + f_s(\pi - \theta)}^2}.
         \end{equation*}
         Mostre que \(\frac13 \geq C \geq -1,\) onde estes limites correspondem aos espalhamentos apenas nos estados de tripleto e de singleto. Dê um exemplo desses limites.
   \end{enumerate}
\end{exercício}
\begin{proof}[Resolução]
   A matriz de espalhamento \(M = \mathcal{M}(\vetor{k}_f,\vetor{k}_i)\) pode ser escrita como
   \begin{equation*}
      M = f_s(k,\theta) \Pi_s + f_t(k,\theta) \Pi_t,
   \end{equation*}
   onde \(\Pi_s\) e \(\Pi_t\) são os projetores ortogonais complementares, \(\Pi_s + \Pi_t = \unity\), dados por
   \begin{equation*}
      \Pi_s = \frac14 \left(\unity - \vetor{\sigma}_n \cdot \vetor{\sigma}_p\right)
      \quad\text{e}\quad
      \Pi_t = \frac14 \left(3\unity + \vetor{\sigma}_n \cdot \vetor{\sigma}_p\right).
   \end{equation*}
   O estado final é dado pela matriz densidade \(\rho_f\) definida por
   \begin{equation*}
      \diff{\sigma}{\Omega}\rho_f = M\rho_i \herm{M} = \left[f_s^* \Pi_s + f_t^* \Pi_t\right]\rho_i\left[f_s^* \Pi_s + f_t^* \Pi_t\right],
   \end{equation*}
   portanto se o estado inicial é não polarizado, \(\rho_i = \frac14,\) temos
   \begin{align*}
      4\diff{\sigma}{\Omega} \rho_f &= \abs{f_s}^2 \Pi_s + \abs{f_t}^2\Pi_t\\
                                    &= \frac{\abs{f_s}^2 + 3\abs{f_t}^2}4 \unity + \frac{\abs{f_t}^2 - \abs{f_s}^2}{4} \vetor{\sigma}_n \cdot \vetor{\sigma}_p.
   \end{align*}
   Assim, a seção de choque diferencial é
   \begin{equation*}
      \diff{\sigma}{\Omega} = \frac14\abs{f_s}^2 + \frac34\abs{f_t}^2
   \end{equation*}
   e há correlação entre os spins do próton e do nêutron dada por
   \begin{equation*}
      C_{ij} = \frac{\abs{f_t}^2 - \abs{f_s}^2}{4 \diff{\sigma}{\Omega}}\delta_{ij} = \frac{\abs{f_t}^2 - \abs{f_s}^2}{3\abs{f_t}^2 + \abs{f_s}^2}\delta_{ij}.
   \end{equation*}

   Para o espalhamento próton-próton, temos
   \begin{equation*}
      M = \left[f_s(\theta) + f_s(\pi - \theta)\right] \Pi_s + \left[f_t(\theta) - f_t(\pi - \theta)\right]\Pi_t
   \end{equation*}
   por conta da simetria de troca, e então obtemos a correlação \(C_{ij} = C \delta_{ij}\)
   \begin{equation*}
      C = \frac{\abs{f_t(\theta) - f_t(\pi - \theta)}^2 - \abs{f_s(\theta) + f_s(\pi - \theta)}^2}{3\abs{f_t(\theta) - f_t(\pi - \theta)}^2 + \abs{f_s(\theta) + f_s(\pi - \theta)}^2}
   \end{equation*}
   analogamente, com a substituição \(f_s \to f_s(\theta) + f_s(\pi - \theta)\) e \(f_t \to f_t(\theta) - f_t(\pi - \theta)\). Notemos que
   \begin{equation*}
       C = -1 + \frac{2\abs{f_t(\theta) - f_t(\pi - \theta)}^2}{3\abs{f_t(\theta) - f_t(\pi - \theta)}^2 + \abs{f_s(\theta) + f_s(\pi - \theta)}^2} \geq -1
   \end{equation*}
   e que
   \begin{equation*}
       C = \frac13 - \frac43\frac{\abs{f_s(\theta) + f_s(\pi - \theta)}^2}{3\abs{f_t(\theta) - f_t(\pi - \theta)}^2 + \abs{f_s(\theta) + f_s(\pi - \theta)}^2} \leq \frac13,
   \end{equation*}
   isto é, 
   \begin{equation*}
      \frac13 \geq C \geq -1
   \end{equation*}
   com as implicações 
   \begin{equation*}
      C = -1 \implies \abs{f_t(\theta) - f_t(\pi-\theta)} = 0
      \quad\text{e}\quad
      C = \frac13 \implies \abs{f_s(\theta) + f_s(\pi-\theta)} = 0.
   \end{equation*}
   Assim, \(C = -1\) e \(C = \frac13\) correspondem ao casos em que há espalhamento apenas nos estados de singleto e tripleto, respectivamente. \todo[Exemplo.]
\end{proof}

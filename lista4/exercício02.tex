% vim: spl=pt
\begin{exercício}{Modelo solúvel}{ex2}
   Mostre que unitariedade e a invariância por reversão temporal levam a \(\abs{S_{aa}} = \abs{S_{bb}}\) para o modelo solúvel dado em aula. A princípio, isso parece errado, já que as seções de choque elástica \(a \to a\) e \(b \to b\) certamente são diferentes por envolverem interações distintas. Resolva o problema de espalhamento com \(b\) incidente e calcule \(S_{bb}(E),\) mostrando que \(S_{aa}\) e \(S_{bb}\) têm de fato o mesmo valor absoluto na mesma energia \(E,\) mas que \(\abs{S_{aa} - 1} \neq \abs{S_{bb} - 1}.\) Finalmente, mostre que a seção de choque total do espalhamento elástico \(b \to b\) é
   \begin{equation*}
      \sigma_{\mathrm{el},bb}(E) = \frac{\pi}{p_b^2} \frac{\left[\Gamma_b(E)\right]^2}{\left[E - E_0 - \Sigma_R(E)\right]^2 + \frac14 \left[\Gamma_{\mathrm{tot}}(E)\right]^2}.
   \end{equation*}
\end{exercício}
\begin{proof}[Resolução]
    
\end{proof}

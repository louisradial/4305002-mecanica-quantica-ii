% vim: spl=pt
\begin{exercício}{Transformação de Foldy-Wouthuysen}{ex2}
    Considere o operador unitário \(U\) e as transformações unitárias \(\tilde{H} = UH \herm{U}\) e \(\tilde{\Psi} = U\Psi\) que mudam a equação de Dirac para
    \begin{equation*}
        i\hbar \diffp{\tilde{\Psi}}{t} = \tilde{H} \tilde{\Psi}.
    \end{equation*}
    Se \(U\) é dado por
    \begin{equation*}
        U = \sqrt{\frac{mc^2 + \abs{E}}{2\abs{E}}} + \frac{\beta \vetor{\alpha}\cdot \vetor{p}}{\sqrt{2\abs{E}(mc^2 + \abs{E})}},
    \end{equation*}
    mostre que \(U\) remove todos os acoplamentos entre as partes de energia positiva e negativa da equação de Dirac. Esse é um exemplo de transformação de Foldy-Wouthuysen.
\end{exercício}
\begin{proof}[Resolução]
    Vamos escrever
    \begin{equation*}
        \cos\theta = \sqrt{m + \abs{E}}{2 \abs{E}}\quad\text{e}\quad
        \sin\theta = \frac{p}{\sqrt{2\abs{E}(m + \abs{E})}}
    \end{equation*}
\end{proof}

% vim: spl=pt
\begin{exercício}{Spinores de Weyl}{ex4}
    Considere o spinor de Dirac \(\Psi(x)\).
    \begin{enumerate}[label=(\alph*)]
        \item Mostre que a corrente axial \(j_5^\mu = \bar{\Psi} \gamma^\mu \gamma_5 \Psi\) é conservada para uma partícula de massa nula. No caso de uma partícula massiva, determine \(\partial_\mu j^\mu_5.\)
        \item Mostre que podemos escrever a equação relativística para uma partícula de massa nula em termos de um spinor de duas componentes \(\chi\) tal que
            \begin{equation*}
                i\hbar (\partial_0 - \vetor{\sigma}\cdot \nabla) \chi = 0.
            \end{equation*}
            Mostre que é possível escrever uma equação para uma partícula massiva com
            \begin{equation*}
                i\hbar \bar{\sigma}^\mu \partial_\mu \chi - imc \sigma^2 \chi^* = 0,
            \end{equation*}
            onde \(\bar{\sigma}^\mu = (\unity, - \vetor{\sigma}).\) Mostre como essa equação se transforma sob transformações de Lorentz e mostre que ela implica a equação de Klein-Gordon.
        \item Escrevendo o spinor de Dirac como
            \begin{equation*}
                \Psi = \begin{pmatrix}
                    \psi_L\\
                    \psi_R
                \end{pmatrix}
            \end{equation*}
            mostre como \(\psi_L\) e \(\psi_R\) se relacionam com as componentes de \(\chi.\)
    \end{enumerate}
\end{exercício}
\begin{proof}[Resolução]
    Para uma solução \(\Psi\) da equação de Dirac, temos
    \begin{align*}
        (i \slashed{\partial} - m) \Psi = 0 &\implies -i \partial_\mu \herm{\Psi} \gamma^0\gamma^\mu\gamma^0 - m \herm{\Psi} = 0\\
                                            &\implies i \partial_\mu \bar{\Psi} \gamma^\mu + m \bar{\Psi} = 0\\
                                            &\implies \bar{\Psi}\overleftarrow{(i \slashed{\partial} + m)} = 0,
    \end{align*}
    então \(\slashed{\partial} \Psi = -im \Psi\) e \(\bar{\Psi} \overleftarrow{\slashed{\partial}} = i m \bar{\Psi}.\) Com isso, definindo a corrente axial \(j_5^\mu = \bar{\Psi} \gamma^\mu \gamma_5 \Psi,\) temos
    \begin{align*}
        \partial_\mu j_5^\mu &= (\partial_\mu \bar{\Psi}) \gamma^\mu \gamma_5 \Psi + \bar{\Psi} \gamma^\mu \gamma_5 \partial_\mu \Psi\\
                             &= \bar{\Psi} \overleftarrow{\slashed{\partial}} \gamma_5 \Psi - \bar{\Psi} \gamma_5 \slashed{\partial} \Psi\\
                             &= i m \bar{\Psi} \gamma_5 \Psi + i m \bar{\Psi} \gamma_5 \Psi\\
                             &= 2im \bar{\Psi} \gamma_5 \Psi\\
                             &= 2im \left(\herm{\Psi}\gamma^0 P_L\Psi_L - \herm{\Psi} \gamma^0 P_R \Psi_R\right)\\
                             &= 2im \left(\herm{\Psi} P_R \gamma^0 \Psi_L - \herm{\Psi} P_L \gamma^0 \Psi_R\right)\\
                             &= 2im \left(\herm{\Psi}_R \gamma^0 \Psi_L - \herm{\Psi}_L \gamma^0 \Psi_R\right)\\
                             &= -2im \herm{\psi}_L \psi_R
    \end{align*}
    então se \(m = 0\) temos \(j_5^\mu\) conservada.
\end{proof}

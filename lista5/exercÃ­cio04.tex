% vim: spl=pt
\begin{exercício}{Spinores de Weyl}{ex4}
    Considere o spinor de Dirac \(\Psi(x)\).
    \begin{enumerate}[label=(\alph*)]
        \item Mostre que a corrente axial \(j_5^\mu = \bar{\Psi} \gamma^\mu \gamma_5 \Psi\) é conservada para uma partícula de massa nula. No caso de uma partícula massiva, determine \(\partial_\mu j^\mu_5.\)
        \item Mostre que podemos escrever a equação relativística para uma partícula de massa nula em termos de um spinor de duas componentes \(\chi\) tal que
            \begin{equation*}
                i\hbar (\partial_0 - \vetor{\sigma}\cdot \nabla) \chi = 0.
            \end{equation*}
            Mostre que é possível escrever uma equação para uma partícula massiva com
            \begin{equation*}
                i\hbar \bar{\sigma}^\mu \partial_\mu \chi - imc \sigma^2 \chi^* = 0,
            \end{equation*}
            onde \(\bar{\sigma}^\mu = (\unity, - \vetor{\sigma}).\) Mostre como essa equação se transforma sob transformações de Lorentz e mostre que ela implica a equação de Klein-Gordon.
        \item Escrevendo o spinor de Dirac como
            \begin{equation*}
                \Psi = \begin{pmatrix}
                    \psi_L\\
                    \psi_R
                \end{pmatrix}
            \end{equation*}
            mostre como \(\psi_L\) e \(\psi_R\) se relacionam com as componentes de \(\chi.\)
    \end{enumerate}
\end{exercício}
\begin{proof}[Resolução]
    
\end{proof}

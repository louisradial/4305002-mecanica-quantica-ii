% vim: spl=pt
\begin{exercício}{Transformação de gauge na equação de Dirac}{ex7}
    Considere a equação de Dirac na presença de um potencial \(A^\mu.\) Mostre que uma transformação de gauge induz uma mudança de fase no espinor de Dirac.
\end{exercício}
\begin{proof}[Resolução]
    Com a substituição mínima \(\partial_\mu \to \partial_\mu - i e A_\mu\) temos
    \begin{equation*}
        (i \slashed{\partial} + e \slashed{A} - m)\Psi(x) = 0
    \end{equation*}
    como a equação de Dirac para uma partícula em um campo eletromagnético. Com a transformação de gauge \(A_\mu \to A_\mu + \partial_\mu \Lambda\) e \(\Psi \to \Psi^\Lambda\), temos
    \begin{equation*}
        (i \slashed{\partial} + e \slashed{A} + e \slashed{\partial}\Lambda - m) \Psi^{\Lambda}(x) = 0.
    \end{equation*}
    Como o gauge escolhido não tem significado físico, temos \(\Psi^\Lambda = U(\Lambda) \Psi\) para algum operador unitário \(U(\Lambda)\), então
    \begin{align*}
        \left[i \slashed{\partial}U(\Lambda) +  i \gamma^\mu U(\Lambda) \partial_\mu + e \slashed{A} U(\Lambda) + e (\slashed{\partial} \Lambda) U(\Lambda) - m U(\Lambda)\right] \Psi(x) = 0.
    \end{align*}
    Se \(\Psi(x)\) é uma solução da equação de Dirac com \(\Lambda = 0\), temos apenas
    \begin{equation*}
        \left\{i \slashed{\partial} U(\Lambda) + (e \slashed{\partial} \Lambda) U(\Lambda) + \commutator{\gamma^\mu}{U(\Lambda)} \left(i\partial_\mu + e A_\mu\right)\right\}\Psi(x) = 0,
    \end{equation*}
    isto é, para que a equação de Dirac seja invariante por transformações de gauge, o operador \(U(\Lambda)\) deve satisfazer
    \begin{equation*}
        i \slashed{\partial} U(\Lambda) + (e \slashed{\partial} \Lambda) U(\Lambda) + \commutator{\gamma^\mu}{U(\Lambda)} \left(i\partial_\mu + e A_\mu\right) = 0
    \end{equation*}
    para qualquer escolha de \(A_\mu\) e \(\Lambda.\) Para que a equação acima seja satisfeita para toda escolha de \(A_\mu,\) devemos ter
    \begin{equation*}
        \commutator{\gamma^\mu}{U(\Lambda)} = 0
    \end{equation*}
    para todo \(\Lambda,\) logo \(U(\Lambda)\) é proporcional à matriz identidade. Assim a condição é reduzida às equações
    \begin{equation*}
        i \partial_\mu U(\Lambda) + e (\partial_\mu \Lambda) U(\Lambda) = 0,
    \end{equation*}
    cuja solução é
    \begin{equation*}
        U(\Lambda) = \exp(i e \Lambda),
    \end{equation*}
    obtendo, portanto, que
    \begin{equation*}
        \Psi^\Lambda(x) = e^{ie \Lambda(x)} \Psi(x)
    \end{equation*}
    é a transformação para o spinor de Dirac.
\end{proof}

% vim: spl=pt
\begin{exercício}{Soluções da equação de Dirac}{ex1}
    Podemos escrever a solução de partícula livre \(u(\vetor{p})\) independente da representação das matrizes \(\gamma^\mu\) como \((\slashed{p} + mc) X_i,\) onde \(X_i\) é um elemento da base canônica. Mostre que \((\slashed{p} + mc) X_i\) é solução de \((\slashed{p} - mc)u(\vetor{p}) = 0\) com \(p^2 = m^2c^4\). Pode parecer que há oito soluções, quatro com \(E > 0\) e outras quatro com \(E < 0.\) Usando uma representação para as matrizes \(\gamma^\mu\), mostre que há apenas quatro soluções independentes e relacione com as soluções determinadas em aula.
\end{exercício}
\begin{proof}[Resolução]
    
\end{proof}

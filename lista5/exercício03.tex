% vim: spl=pt
\begin{exercício}{Representação de Majorana}{ex3}
    Uma representação de Majorana é uma representação em que todas as matrizes \(\gamma^\mu\) são imaginárias e as matrizes \(\vetor{\alpha}\) são simétricas. Encontre uma forma explícita das matrizes de Dirac nessa representação. Mostre que nessa representação a equação de Dirac é real. Encontre, nesse caso, soluções para a equação de Dirac que satisfaçam a condição de Majorana \(\Psi = \Psi^*\).
\end{exercício}
\begin{proof}[Resolução]
    Consideramos a matriz
    \begin{equation*}
        U = \gamma^0\frac{\unity + \gamma^2}{\sqrt{2}}
    \end{equation*}
    que é hermitiana,
    \begin{equation*}
        \herm{U} = \left(\frac{\unity + \gamma^2}{\sqrt{2}}\right)^\dagger \gamma^0 = \gamma^0\frac{\unity + \gamma^2}{\sqrt{2}}\gamma^0 \gamma^0 = \gamma^0 \frac{\unity + \gamma^2}{\sqrt{2}} = U
    \end{equation*}
    e é unitária,
    \begin{equation*}
        \herm{U}U = \frac12 (\gamma^0 + \gamma^0 \gamma^2)^2 = \frac12 (\unity + \gamma^2 + \gamma^0 \gamma^2 \gamma^0 + \gamma^0 \gamma^2 \gamma^0 \gamma^2) = \unity.
    \end{equation*}
    Definindo \(\tilde{\gamma}^\mu = U \gamma^\mu \herm{U}\), temos
    \begin{align*}
        \tilde{\gamma}^\mu &= \frac12 \gamma^0 (\unity + \gamma^2) \gamma^\mu (\unity - \gamma^2) \gamma^0\\
                           &= \frac12 \gamma^0 (\gamma^\mu + \gamma^2 \gamma^\mu - \gamma^\mu \gamma^2 - \gamma^2 \gamma^\mu \gamma^2) \gamma^0\\
                           &= \frac12 \gamma^0 (\gamma^\mu + \commutator{\gamma^2}{\gamma^\mu} + \gamma^2 \gamma^2 \gamma^\mu - \gamma^2 \anticommutator{\gamma^2}{\gamma^\mu})\gamma^0\\
                           &= \gamma^0 (\gamma^2 \gamma^\mu - g^{\mu 2} \unity - g^{\mu 2}\gamma^2)\gamma^0\\
                           &= \gamma^0 \gamma^2 \gamma^\mu \gamma^0 - g^{\mu2} (\unity - \gamma^2)\\
                           &= \begin{cases}
                               -\gamma^\mu,&\mu = 2\\
                               \gamma^0 \gamma^2 \gamma^\mu\gamma^0,&\mu \neq 2
                           \end{cases}
    \end{align*}
    e \(\anticommutator{\tilde{\gamma}^\mu}{\tilde{\gamma}^\nu} = 2g^{\mu\nu}\) já que \(U\) é não singular. Assim, partindo da representação de Dirac,
    \begin{equation*}
        \gamma^0 = \begin{pmatrix}
            \unity &&\\
                   && -\unity
        \end{pmatrix}
        \quad\text{e}\quad
        \gamma^i = \begin{pmatrix}
            && \sigma^i\\
            -\sigma^i &&
        \end{pmatrix}
    \end{equation*}
    obtemos
    \begin{equation*}
        \tilde{\gamma}^0 
        % = \gamma^0 \gamma^2 
        = \begin{pmatrix}
            && \sigma^2\\
            \sigma^2 &&
        \end{pmatrix},\quad
        \tilde{\gamma}^1 
        % = \gamma^2 \gamma^1 
        = \begin{pmatrix}
            i \sigma^3 && \\
            && i \sigma^3
        \end{pmatrix},\quad
        \tilde{\gamma}^2
        % = -\gamma^2
        = \begin{pmatrix}
            && -\sigma^2\\
            \sigma^2 &&
        \end{pmatrix},\quad\text{e}\quad
        \tilde{\gamma}^3 
        % = \gamma^2 \gamma^3 
        = \begin{pmatrix}
            -i \sigma^1 && \\
            && -i \sigma^1
        \end{pmatrix}
    \end{equation*}
    como a representação de Majorana.

    Nessa representação, a equação de Dirac é dada por
    \begin{equation*}
        (i \slashed{\partial} - m) \Psi = 0 \implies \begin{pmatrix}
            \sigma^3 \partial_1 - \sigma^1 \partial_3 - m && i\sigma^2 \partial_0 - i\sigma^2 \partial_2\\
            i \sigma^2 \partial_0 + i \sigma^2 \partial_2 && \sigma^3 \partial_1 - \sigma^1 \partial_3 - m
        \end{pmatrix}
        \Psi(x) = 0,
    \end{equation*}
    portanto é uma equação real. Se \(\Psi\) é solução da equação de Dirac na representação de Dirac, então \(U\Psi\) é solução da equação de Dirac na representação de Majorana. Para um spinor \((\begin{smallmatrix} \varphi\\\chi \end{smallmatrix})\) na representação de Dirac, temos
    \begin{equation*}
        U \begin{pmatrix}
            \varphi\\\chi
        \end{pmatrix} = \gamma^0\frac{\unity + \gamma^2}{\sqrt{2}} \begin{pmatrix}
            \varphi\\\chi
        \end{pmatrix} = \frac{1}{\sqrt{2}} \begin{pmatrix}
            \varphi + \sigma^2 \chi\\
            \sigma^2\varphi - \chi
        \end{pmatrix}
    \end{equation*}
    na representação de Majorana, portanto temos as soluções de energia positiva
    \begin{equation*}
        \tilde{u}_{(1)}(p) = \sqrt{\frac{m + p_0}2}  \begin{pmatrix}
            \xi_+ + \sigma^2\frac{\vetor{\sigma}\cdot\vetor{p}}{p_0 + m}\xi_- \\
            \sigma^2\xi_+ - \frac{\vetor{\sigma}\cdot\vetor{p}}{p_0 + m} \xi_+
        \end{pmatrix}
        \quad\text{e}\quad
        \tilde{u}_{(2)} = \sqrt{\frac{m + p_0}2} \begin{pmatrix}
            \xi_- + \sigma^2 \frac{\vetor{\sigma}\cdot\vetor{p}}{p_0 + m} \xi_-\\
            \sigma^2\xi_- - \frac{\vetor{\sigma}\cdot\vetor{p}}{p_0 + m} \xi_-
        \end{pmatrix}
    \end{equation*}
    \begin{equation*}
        \tilde{u}_{(1)} = 
    \end{equation*}
    \begin{equation*}
        u_{(3)} = \sqrt{m - p_0}\begin{pmatrix}
            \frac{\vetor{\sigma}\cdot\vetor{p}}{p_0 - m} \xi_+\\
            \xi_+
        \end{pmatrix}
        \quad\text{e}\quad
        u_{(4)} = \sqrt{m - p_0}\begin{pmatrix}
            \frac{\vetor{\sigma}\cdot\vetor{p}}{p_0 - m} \xi_-\\
            \xi_-
        \end{pmatrix}.
    \end{equation*}
    O caso de energia positiva é obtido analogamente, obtendo as soluções independentes
    satisfazendo a mesma condição de normalização.
\end{proof}

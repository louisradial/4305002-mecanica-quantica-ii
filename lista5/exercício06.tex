% vim: spl=pt
\begin{exercício}{Álgebra de Clifford e \(\mathrm{SO}(n)\)}{ex6}
    Dado um vetor \(V_i,\) com \(i \in \set{1,2,\dots, n},\) definimos \(n\) matrizes \(\gamma\) por \(V = V_i \gamma_i\) e \(V^2 = (V_i V_i) \unity,\) o que determina uma álgebra de Clifford.
    \begin{enumerate}[label=(\alph*)]
        \item Quais são as condições para as matrizes \(\gamma\) desta álgebra de Clifford.
        \item Assumindo que uma rotação pode ser descrita por \(V \to \herm{\Omega}V \Omega,\) ainda vale \(V^2 = V_i V_i \unity?\) Essa transformação determina a representação spinorial de \(\mathrm{SO}(n)\).
        \item Escrevendo \(\Omega = \unity + \frac{i}{2}\omega_{ik}J^{S}_{ik}\) onde \(J^{S}_{ik}\) são os geradores da representação spinorial, mostre que se identificarmos \(J^{S}_{ik} = \frac{i}{4}\commutator{\gamma_i}{\gamma_k}\), obtemos a álgebra correta para \(\mathrm{SO}(4).\)
        \item Mostre que no caso de \(\mathrm{SO}(3)\) temos \(J^{S}_i = \frac{\sigma_i}{2}\).
        \item Construa as matrizes \(\gamma\) no caso de \(\mathrm{SO}(4)\).
    \end{enumerate}
\end{exercício}
\begin{proof}[Resolução]
    
\end{proof}

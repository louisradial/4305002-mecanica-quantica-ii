% vim: spl=pt
\begin{exercício}{Álgebra de Clifford e \(\mathrm{SO}(n)\)}{ex6}
    Dado um vetor \(V_i,\) com \(i \in \set{1,2,\dots, n},\) definimos \(n\) matrizes \(\gamma\) por \(V = V_i \gamma_i\) e \(V^2 = (V_i V_i) \unity,\) o que determina uma álgebra de Clifford.
    \begin{enumerate}[label=(\alph*)]
        \item Quais são as condições para as matrizes \(\gamma\) desta álgebra de Clifford.
        \item Assumindo que uma rotação pode ser descrita por \(V \to \herm{\Omega}V \Omega,\) ainda vale \(V^2 = V_i V_i \unity?\) Essa transformação determina a representação spinorial de \(\mathrm{SO}(n)\).
        \item Escrevendo \(\Omega = \unity + \frac{i}{2}\omega_{ik}J^{S}_{ik}\) onde \(J^{S}_{ik}\) são os geradores da representação spinorial, mostre que se identificarmos \(J^{S}_{ik} = \frac{i}{4}\commutator{\gamma_i}{\gamma_k}\), obtemos a álgebra correta para \(\mathrm{SO}(4).\)
        \item Mostre que no caso de \(\mathrm{SO}(3)\) temos \(J^{S}_i = \frac{\sigma_i}{2}\).
        \item Construa as matrizes \(\gamma\) no caso de \(\mathrm{SO}(4)\).
    \end{enumerate}
\end{exercício}
\begin{proof}[Resolução]
    Por motivos psicológicos, tomamos os índices no conjunto \(\mu \in \set{0,1, \dots, n-1}\) e os índices das matrizes \(\gamma\) com índices em cima, \(\gamma^\mu\), e com os índices dos vetores embaixo, \(V_\mu.\) Ainda, quando os índices percorrem apenas o conjunto \(\set{1,\dots, n-1}\) utilizaremos letras do alfabeto romano.

    Para que \(V^2 = (V_\mu V_\nu g^{\mu\nu})\unity\) devemos ter \(\anticommutator{\gamma^\mu}{\gamma^\nu} = 2 g^{\mu \nu} \unity,\) já que
    \begin{equation*}
        0 = V^2 - V^2 = V_\mu \gamma^\mu V_\nu \gamma^\nu - V_\mu V_\nu g^{\mu\nu}\unity = V_\mu V_\nu (\gamma^\mu \gamma^\nu - g^{\mu\nu} \unity) = 2V_\mu V_\nu (\anticommutator{\gamma^\mu}{\gamma^\nu} - 2 g^{\mu\nu} \unity),
    \end{equation*}
    onde usamos que \(V_\mu V_\nu\) é simétrico em relação aos índices. Sob uma rotação \(\tilde{V} = \herm{\Omega} V \Omega\), temos
    \begin{equation*}
        \tilde{V}^2 = \herm{\Omega} V \Omega \herm{\Omega} V \Omega = \herm{\Omega} V^2 \Omega = (V_\mu V_\nu g^{\mu\nu}) \herm{\Omega} \unity \Omega = V_\mu V_\nu g^{\mu\nu} \unity = V^2,
    \end{equation*}
    portanto a álgebra de Clifford mantém a norma invariante sob essas transformações.

    No caso de \(\mathrm{SO}(4)\) identificamos os geradores \(J^{\mu\nu} = \frac{i}{4} \commutator{\gamma^\mu}{\gamma^\nu}\) e definimos
    \begin{equation*}
        J^k = \frac12\epsilon\indices{^k_{ij}} J^{i j} = \frac{i}{8} \epsilon\indices{^k_{ij}} \commutator{\gamma^i}{\gamma^j} = \frac{i}{4} \epsilon\indices{^k_{ij}} \gamma^i \gamma^j
        \quad\text{e}\quad
        K^k = J^{0k} = \frac{i}{4} \commutator{\gamma^0}{\gamma^k} = \frac{i}{2} \gamma^0 \gamma^k.
    \end{equation*}
    Assim, \(\commutator{\gamma^i}{\gamma^j} = \frac{4}{i} \epsilon\indices{^{ij}_k} J^k,\) \(\gamma^0 \gamma^k = \frac{2}{i} K^k,\) e então temos
    \begin{align*}
        \commutator{J^i}{J^j} &= -\frac{1}{16} \epsilon\indices{^i_{ab}} \epsilon\indices{^j_{mn}} \commutator{\gamma^a \gamma^b}{\gamma^m \gamma^n}\\
                              &= - \frac{1}{16} \epsilon\indices{^i_{ab}} \epsilon\indices{^j_{mn}} \left(\gamma^a \commutator{\gamma^b}{\gamma^m\gamma^n} + \commutator{\gamma^a}{\gamma^m\gamma^n}\gamma^b\right)\\
                              &= - \frac{1}{16} \epsilon\indices{^i_{ab}} \epsilon\indices{^j_{mn}} \left(\gamma^a \gamma^m\anticommutator{\gamma^b}{\gamma^n} - \gamma^a \anticommutator{\gamma^b}{\gamma^m} \gamma^n + \gamma^m \anticommutator{\gamma^a}{\gamma^n} \gamma^b - \anticommutator{\gamma^a}{\gamma^m} \gamma^n \gamma^b\right)\\
                              &= - \frac{1}{8} \epsilon\indices{^i_{ab}} \epsilon\indices{^j_{mn}}\left(g^{bn} \gamma^a \gamma^m - g^{bm} \gamma^a \gamma^n + g^{an} \gamma^m \gamma^b - g^{am} \gamma^n \gamma^b\right)\\
                              &= - \frac14 \left(\epsilon\indices{^i_{ab}} \epsilon\indices{^{bj}_m} \gamma^a \gamma^m - \epsilon\indices{^i_{ba}} \epsilon\indices{^{aj}_m} \gamma^m \gamma^b\right)\\
                              &= -\frac14 \left(\delta^{ij} \delta_{am}\gamma^a\gamma^m - \delta\indices{^i_m}\delta\indices{^j_a}\gamma^a \gamma^m - \delta^{ij} \delta_{bm} \gamma^m \gamma^b + \delta\indices{^i_m}\delta\indices{^j_b} \gamma^m \gamma^b\right)\\
                              &= -\frac14\left(\gamma^i \gamma^j - \gamma^j \gamma^i\right)\\
                              &= -\frac14 \commutator{\gamma^i}{\gamma^j}\\
                              &= i \epsilon\indices{^{ij}_k}J^k,
    \end{align*}
    \begin{align*}
        \commutator{J^i}{K^j} &= -\frac{1}{8} \epsilon\indices{^i_{ab}} \commutator{\gamma^a \gamma^b}{\gamma^0\gamma^j}\\
                              &= -\frac{1}{8} \epsilon\indices{^i_{ab}} \left(\gamma^a \gamma^b \gamma^0 \gamma^j - \gamma^0 \gamma^j \gamma^a \gamma^b\right)\\
                              &= -\frac{1}{8} \epsilon\indices{^i_{ab}} \gamma^0\commutator{\gamma^a \gamma^b}{\gamma^j}\\
                              &= -\frac{1}{8}\epsilon\indices{^i_{ab}} \gamma^0\left(\gamma^a\anticommutator{\gamma^j}{\gamma^b} - \anticommutator{\gamma^j}{\gamma^a} \gamma^b\right)\\
                              &= -\frac{1}{4}\epsilon\indices{^i_{ab}} \gamma^0\left(g^{jb}\gamma^a- g^{ja} \gamma^b\right)\\
                              &= -\frac12 \epsilon\indices{^{ij}_a} \gamma^0 \gamma^a\\
                              &= i\epsilon\indices{^{ij}_k}K^k,
    \end{align*}
    e
    \begin{equation*}
        \commutator{K^i}{K^j} = -\frac{1}{4}\commutator{\gamma^0 \gamma^i}{\gamma^0\gamma^j}
        = -\frac{1}{4}\left(\gamma^0 \gamma^i \gamma^0 \gamma^j - \gamma^0 \gamma^j \gamma^0 \gamma^i\right)
        = \frac{1}{4}\gamma^0 \gamma^0 \commutator{\gamma^i}{\gamma^j}
        = -i\epsilon\indices{^{ij}_k} J^k.
    \end{equation*}
    Resumindo, temos seis geradores \(\vetor{J}\) e \(\vetor{K}\) que satisfazem a álgebra
    \begin{equation*}
        \commutator{J^i}{J^j} = i \epsilon\indices{^{ij}_k} J^k,\quad
        \commutator{J^i}{K^j} = i \epsilon\indices{^{ij}_k} K^k,\quad\text{e}\quad
        \commutator{K^i}{K^j} = -i \epsilon\indices{^{ij}_k} J^k,
    \end{equation*}
    que é justamente a álgebra de Lie \(\mathfrak{so}(4).\)

    Para \(\mathrm{SO}(3)\) 
\end{proof}

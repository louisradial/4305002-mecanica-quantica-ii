% vim: spl=pt
\begin{exercício}{Operadores de aniquilação e criação fermiônicos em sistema de dois níveis}{ex1}
    Construa explicitamente as matrizes \(4\times4\) que representam os operadores de aniquilação e criação fermiônicos \(a_0, \herm{a}_0, a_1, \herm{a}_1\) para um sistema de dois níveis e verifique as relações de anticomutação.
\end{exercício}
\begin{proof}[Resolução]
    Consideramos a base \(\set{\ket{00}, \ket{01}, \ket{10}, \ket{11}},\) em que \(\ket{n_0 n_1}\) é autovetor de \(\herm{a}_0a_0\) associado ao autovalor \(n_0 \in \set{0,1}\) e autovetor de \(\herm{a}_1 a_1\) associado ao autovalor \(n_1 \in \set{0,1}.\) De
    \begin{equation*}
        a_0 \ket{00} = 0,\quad
        a_0 \ket{01} = 0,\quad
        a_0 \ket{10} = \ket{00},\quad\text{e}\quad
        a_0 \ket{11} = -\ket{01}
    \end{equation*}
    temos a representação
    \begin{equation*}
        a_0 \doteq \begin{pmatrix}
            && \sigma^3\\
            &&
        \end{pmatrix}.
    \end{equation*}
    De 
    \begin{equation*}
        \herm{a}_0 \ket{00} = \ket{10},\quad
        \herm{a}_0 \ket{01} = -\ket{11},\quad
        \herm{a}_0 \ket{10} = 0,\quad\text{e}\quad
        \herm{a}_0 \ket{11} = 0
    \end{equation*}
    temos a representação
    \begin{equation*}
        \herm{a}_0 \doteq \begin{pmatrix}
        && \\
            \sigma^3 &&
        \end{pmatrix}
    \end{equation*}
    e então
    \begin{equation*}
        \herm{a}_0a_0 \doteq \begin{pmatrix}
            &&\\&&\unity
        \end{pmatrix}
        \quad\text{e}\quad
        \anticommutator{a_0}{\herm{a}_0} \doteq \begin{pmatrix}
            \unity && \\
                   && \unity
        \end{pmatrix}.
    \end{equation*}
    De
    \begin{equation*}
        a_1\ket{00} = 0,\quad
        a_1\ket{01} = \ket{00},\quad
        a_1\ket{10} = 0,\quad\text{e}\quad
        a_1\ket{11} = \ket{10},
    \end{equation*}
    temos a representação
    \begin{equation*}
        a_1 \doteq \begin{pmatrix}
            0 && 1 && 0 && 0\\
            0 && 0 && 0 && 0\\
            0 && 0 && 0 && 1\\
            0 && 0 && 0 && 0
            \end{pmatrix} = \begin{pmatrix}
            \sigma^+ &&\\
                            &&\sigma^+
        \end{pmatrix},
    \end{equation*}
    onde \(\sigma^+ = \frac12 (\sigma^1 +i \sigma^2).\) De
    \begin{equation*}
        \herm{a}_1\ket{00} = \ket{01},\quad
        \herm{a}_1\ket{01} = 0,\quad
        \herm{a}_1\ket{10} = \ket{11},\quad\text{e}\quad
        \herm{a}_1\ket{11} = 0,
    \end{equation*}
    temos a representação
    \begin{equation*}
        \herm{a}_1 \doteq \begin{pmatrix}
            0 && 0 && 0 && 0\\
            1 && 0 && 0 && 0\\
            0 && 0 && 0 && 0\\
            0 && 0 && 1 && 0
            \end{pmatrix} = \begin{pmatrix}
            \sigma^- &&\\
                            &&\sigma^-
        \end{pmatrix},
    \end{equation*}
    onde \(\sigma^- = \frac12 (\sigma^1 - i \sigma^2),\) e então
    \begin{equation*}
        \herm{a}_1a_1 \doteq \begin{pmatrix}
            \sigma^-\sigma^+ &&\\&&\sigma^- \sigma^+
        \end{pmatrix}
        \quad\text{e}\quad
        \anticommutator{a_1}{\herm{a}_1} \doteq \begin{pmatrix}
            \unity && \\
                   && \unity
        \end{pmatrix},
    \end{equation*}
    onde \(\sigma^- \sigma^+ = \frac12 (\unity - \sigma^3) = (\begin{smallmatrix} 0&&0\\0&& 1 \end{smallmatrix})\).

    Na base escolhida, determinamos as representações
    \begin{equation*}
        a_0 \doteq \sigma^+ \otimes \sigma^3,\quad
        \herm{a}_0 \doteq \sigma^- \otimes \sigma^3,\quad
        a_1 \doteq \unity \otimes \sigma^+,\quad\text{e}\quad
        \herm{a}_1 \doteq \unity \otimes \sigma^-
    \end{equation*}
    e as relações de anticomutação
    \begin{equation*}
        \anticommutator{a_0}{\herm{a}_0} \doteq \unity \otimes \unity \doteq \anticommutator{a_1}{\herm{a}_1}.
    \end{equation*}
    Para as demais, temos
    \begin{align*}
        \anticommutator{a_0}{a_0} &= 2 \sigma^+ \sigma^+ \otimes \unity&
        \anticommutator{a_0}{a_1} &= \sigma^+\anticommutator{\sigma^3}{\sigma^-}&
        \anticommutator{a_0}{\herm{a}_1} &= \sigma^+ \otimes \anticommutator{\sigma^3}{\sigma^-}&
        \anticommutator{a_1}{a_1} &= 2\unity \otimes \sigma^+ \sigma^+\\
                                  &= 0&
                                  &= 0&
                                  &= 0&
                                  &= 0
    \end{align*}
    e as outras podem ser obtidas a partir destas, utilizando \(\anticommutator{x}{y}^\dag = \anticommutator{x^\dag}{y^\dag}.\)
\end{proof}

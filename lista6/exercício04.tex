% vim: spl=pt
\begin{exercício}{Transformação de Bogoliubov}{ex4}
    Considere operadores de aniquilação e criação que satisfazem as relaçõse de comutação canônica \(\commutator{a}{\herm{a}} = \unity\).
    \begin{enumerate}[label=(\alph*)]
        \item Mostre que a transformação de Bogoliubov
            \begin{equation*}
                b = a \cosh \eta + \herm{a} \sinh \eta
            \end{equation*}
            preserva a relação de comutação, isto é, \(\commutator{b}{\herm{b}} = \unity.\)
        \item Utilizando uma transformação de Bogoliubov, determine os autovalores do hamiltoniano
            \begin{equation*}
                H = \hbar \omega \herm{a} a + \frac12 V (aa + \herm{a} \herm{a}).
            \end{equation*}
            Existe um limite superior para \(V\) que permite isso.
        \item Mostre que o operador unitário
            \begin{equation*}
                U = \exp\left[\frac{\eta}{2} (aa - \herm{a} \herm{a})\right]
            \end{equation*}
            satisfaz \(b = U a \herm{U}.\)
        \item Escreva o estado fundamental do hamiltoniano \(H\) em termos dos autoestados do operador de número \(\herm{a} a\).
    \end{enumerate}
\end{exercício}
\begin{proof}[Resolução]
    
\end{proof}

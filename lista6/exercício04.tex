% vim: spl=pt
\begin{exercício}{Transformação de Bogoliubov}{ex4}
    Considere operadores de aniquilação e criação que satisfazem as relaçõse de comutação canônica \(\commutator{a}{\herm{a}} = \unity\).
    \begin{enumerate}[label=(\alph*)]
        \item Mostre que a transformação de Bogoliubov
            \begin{equation*}
                b = a \cosh \eta + \herm{a} \sinh \eta
            \end{equation*}
            preserva a relação de comutação, isto é, \(\commutator{b}{\herm{b}} = \unity.\)
        \item Utilizando uma transformação de Bogoliubov, determine os autovalores do hamiltoniano
            \begin{equation*}
                H = \hbar \omega \herm{a} a + \frac12 V (aa + \herm{a} \herm{a}).
            \end{equation*}
            Existe um limite superior para \(V\) que permite isso.
        \item Mostre que o operador unitário
            \begin{equation*}
                U = \exp\left[\frac{\eta}{2} (aa - \herm{a} \herm{a})\right]
            \end{equation*}
            satisfaz \(b = U a \herm{U}.\)
        \item Escreva o estado fundamental do hamiltoniano \(H\) em termos dos autoestados do operador de número \(\herm{a} a\).
    \end{enumerate}
\end{exercício}
\begin{proof}[Resolução]
    Consideramos o conjunto \(S = \setc{(z, w) \in \mathbb{C}^2}{\abs{z}^2 - \abs{w}^2 = 1}\) que mantém a relação de comutação \(\commutator{a}{\herm{a}}\) invariante sob transformações da forma \(a \to \tilde{a} = \alpha a + \beta \herm{a}\) sempre que \((\alpha, \beta) \in S.\) De fato, temos
    \begin{equation*}
        \commutator{\tilde{a}}{\herm{\tilde{a}}} = \commutator{\alpha a + \beta \herm{a}}{\alpha^* \herm{a} + \beta^* a} = \left(\abs{\alpha}^2 - \abs{\beta}^2\right)\commutator{a}{\herm{a}}
    \end{equation*}
    e então se \((\alpha, \beta) \in S\) obtemos \(\commutator{\tilde{a}}{\herm{\tilde{a}}} = \commutator{a}{\herm{a}}.\) Um subconjunto de \(S\) é a hipérbole
    \begin{equation*}
        H= \setc{(\cosh\eta, \sinh\eta)}{\eta \in \mathbb{R}} \subset S
    \end{equation*}
    e então a transformação de Bogoliubov
    \begin{equation*}
        a \to b = a \cosh \eta + \herm{a} \sinh \eta,
    \end{equation*}
    satisfaz \(\commutator{b}{\herm{b}} = \commutator{a}{\herm{a}}.\)

    Notemos que
    \begin{align*}
        \herm{b}b &= \left(\herm{a} \cosh \eta + a \sinh\eta\right)\left(a \cosh \eta + \herm{a} \sinh \eta\right)\\
                  &= \herm{a} a \cosh^2\eta + (\herm{a}\herm{a} + aa)\cosh\eta \sinh\eta + a \herm{a} \sinh^2\eta\\
                  &= \herm{a} a (\cosh^2 \eta + \sinh^2 \eta) + \sinh^2 \eta + \cosh\eta \sinh\eta (aa + \herm{a} \herm{a})\\
                  &= \herm{a}a \cosh2\eta + \frac12 \sinh2\eta (aa + \herm{a}\herm{a}) + \sinh^2\eta
    \end{align*}
    Considerando o Hamiltoniano \(H = \hbar \omega \herm{a} a + \frac12 V (aa + \herm{a} \herm{a}),\) definimos \(\eta\) a partir de
    \begin{equation*}
        \cosh^2\eta + \sinh^2 \eta = \frac{\hbar\omega}{V}\quad\text{e}\quad 2\sinh\eta \cosh\eta = 1 \iff
        \sinh^2 \eta = \frac{\hbar \omega - V}{2V}\quad\text{e}\quad \sinh 2\eta = 1
    \end{equation*}
    e obtemos
    \begin{align*}
        H &= \hbar \omega \herm{a} a + \frac12 V (aa + \herm{a} \herm{a})\\
          &= V \left[\frac{\hbar\omega}{V} \herm{a} a + \frac12 (aa + \herm{a} \herm{a})\right]\\
          &= V \left[(\cosh^2\eta + \sinh^2 \eta) \herm{a} a + \sinh \eta \cosh \eta (aa + \herm{a} \herm{a})\right]\\
          &= V \left[ \herm{b} b - \sinh^2 \eta\right]\\
          &= V \left(\herm{b} b  + \frac12\right) - \frac12 \hbar \omega,
    \end{align*}
    portanto o espectro de \(H\) é \(\sigma(H) = \setc{(n + \frac12) V - \frac12 \hbar \omega}{n \in \mathbb{N}_0}.\) Nas definições, utilizamos que \(\frac{\hbar \omega}{V} - 1 \geq 0\) para definir \(\sinh^2\eta,\) logo a transformação utilizada só é válida para \(V \leq \hbar \omega.\)

    Seja o operador unitário
    \begin{equation*}
        U(\eta) = \exp\left[\frac{\eta}{2}\right]
    \end{equation*}
\end{proof}

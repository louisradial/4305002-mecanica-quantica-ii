% vim: spl=pt
\begin{exercício}{Representação de Schwinger para momento angular}{ex6}
    Na representação de boson de Schwinger, o operador de spin é representado por dois operadores bosônicos \(a\) e \(b\) com
    \begin{equation*}
        S_+ = \herm{a} b,\quad S_- = \herm{S}_+,\quad\text{e}\quad S_z = \frac12 (\herm{a} a - \herm{b}b).
    \end{equation*}
    \begin{enumerate}[label=(\alph*)]
        \item Mostre que essas definições são consistentes com as relações de comutação de momento angular.
        \item Usando as relações de comutação bosônicas, mostre que
            \begin{equation*}
                \ket{sm} = \frac{(\herm{a})^{s + m}}{\sqrt{(s+m)!}}\frac{(\herm{b})^{s - m}}{\sqrt{(s-m)!}} \ket{\Omega}
            \end{equation*}
            é compatível com a definição de um autovetor simultâneo de \(\vetor{S}^2\) e \(S_z.\) Aqui \(\ket{\Omega}\) é o vácuo para os bósons de Schwinger, e o spin \(s\) define um subespaço \(\setc{\ket{n_an_b}}{n_a + n_b = 2s}.\)
    \end{enumerate}
\end{exercício}
\begin{proof}[Resolução]
    
\end{proof}

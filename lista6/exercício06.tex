% vim: spl=pt
\begin{exercício}{Representação de Schwinger para momento angular}{ex6}
    Na representação de boson de Schwinger, o operador de spin é representado por dois operadores bosônicos \(a\) e \(b\) com
    \begin{equation*}
        S_+ = \herm{a} b,\quad S_- = \herm{S}_+,\quad\text{e}\quad S_z = \frac12 (\herm{a} a - \herm{b}b).
    \end{equation*}
    \begin{enumerate}[label=(\alph*)]
        \item Mostre que essas definições são consistentes com as relações de comutação de momento angular.
        \item Usando as relações de comutação bosônicas, mostre que
            \begin{equation*}
                \ket{sm} = \frac{(\herm{a})^{s + m}}{\sqrt{(s+m)!}}\frac{(\herm{b})^{s - m}}{\sqrt{(s-m)!}} \ket{\Omega}
            \end{equation*}
            é compatível com a definição de um autovetor simultâneo de \(\vetor{S}^2\) e \(S_z.\) Aqui \(\ket{\Omega}\) é o vácuo para os bósons de Schwinger, e o spin \(s\) define um subespaço \(\setc{\ket{n_an_b}}{n_a + n_b = 2s}.\)
    \end{enumerate}
\end{exercício}
\begin{proof}[Resolução]
    Comparando com a base esférica, escrevemos \(S_{+1} = -2^{-\frac12}S_+,\) \(S_{-1} = 2^{-\frac12}S_-,\) e \(S_0 = S_z\)  e para verificar a consistência com as relações de comutação de momento angular, basta determinarmos que \(\commutator{S_0}{S_{+1}} = + S_{+1},\) já que
    \begin{equation*}
        \commutator{S_0}{S_{-1}} = - \commutator{S_0}{\herm{S}_{+1}} = - \commutator{S_{+1}}{S_0}^\dag = \herm{S}_{+1} = - S_{-1}
    \end{equation*}
    segue como resultado. Temos
    \begin{align*}
        \commutator{S_0}{S_{+1}} &= -2^{-\frac12}\commutator{S_0}{\herm{a} b}\\
                                 &= \frac12 2^{-\frac12}\left(\herm{a}\commutator{b}{S_0}-  \commutator{S_0}{\herm{a}}b\right)\\
                                 &= \frac12 2^{-\frac12}\left(\herm{a}\diffp{S_0}{\herm{b}} - \diffp{S_0}{a} b\right)\\
                                 &= -2^{-\frac12} \herm{a}b\\
                                 &= S_{+1}
    \end{align*}
    confirmando a consistência. Em termos dos operadores bosônicos, temos
    \begin{align*}
        \vetor{S}^2 &= S_0^2 - S_{-1} S_{+1} - S_{+1} S_{-1}\\
                    &= \frac14 (\herm{a}a)^2  + \frac14 (\herm{b}b)^2 - \frac12\herm{a}a \herm{b} b + \frac12\herm{a}b \herm{b} a + \frac12\herm{b} a \herm{a} b\\
                    &= \frac14 (\herm{a} a)^2 + \frac14 (\herm{b} b)^2 + \frac12\herm{a}a\herm{b}b + \frac12 \herm{a} a + \frac12 \herm{b} b\\
                    &= \frac{\herm{a} a + \herm{b} b}{2} \left(\frac{\herm{a} a + \herm{b} b}{2} + \unity\right)
    \end{align*}
    portanto o subespaço \(\setc{\ket{n_an_b}}{n_a + n_b = 2s}\) é o subespaço associado ao autovalor \(s(s + 1)\) de \(\vetor{S}^2.\)

    Consideramos agora os estados da forma
    \begin{equation*}
        \ket{sm} = \frac{(\herm{a})^{s+m}}{\sqrt{(s+m)!}} \frac{(\herm{b})^{s-m}}{\sqrt{(s - m)!}} \ket{\Omega}
    \end{equation*}
    e desejamos mostrar que
    \begin{equation*}
        \vetor{S}^2 \ket{sm} = s(s+1) \ket{sm}\quad\text{e}\quad S_z \ket{sm} = m \ket{sm},
    \end{equation*}
    onde \(s\) define o subespaço \(\setc{\ket{n_an_b}}{n_a + n_b = 2s}.\) Notemos que
    \begin{align*}
        \herm{a}a \ket{sm} &= \herm{a} a \frac{(\herm{a})^{s+m}}{\sqrt{(s+m)!}} \frac{(\herm{b})^{s-m}}{\sqrt{(s - m)!}} \ket{\Omega}\\
                           &= \herm{a} \diffp*{\left[\frac{(\herm{a})^{s+m}}{\sqrt{(s+m)!}}\frac{(\herm{b})^{s-m}}{\sqrt{(s - m)!}}\right]}{\herm{a}} \ket{\Omega} + \herm{a}\frac{(\herm{a})^{s+m}}{\sqrt{(s + m)!}}\frac{(\herm{b})^{s-m}}{\sqrt{(s - m)!}}a\ket{\Omega}\\
                           &= (s + m) \frac{(\herm{a})^{s+m}}{\sqrt{(s+m)!}} \frac{(\herm{b})^{s-m}}{\sqrt{(s - m)!}} \ket{\Omega}\\
                           &= (s + m)\ket{sm},
    \end{align*}
    e, analogamente, \(\herm{b}b \ket{sm} = (s-m) \ket{sm},\) portanto
    \begin{align*}
        S_z \ket{sm} &= \frac{\herm{a}a - \herm{b} b}{2}\ket{sm}&
        \vetor{S}^2\ket{sm} &= \frac{\herm{a}a + \herm{b}b}{2}\left(\frac{\herm{a}a + \herm{b}b}{2} + \unity\right)\ket{sm}\\
                            &= \frac{(s + m) - (s - m)}{2} \ket{sm}&
                            &= \frac{(s + m) + (s - m)}{2}\left(\frac{(s + m) + (s - m)}{2} + 1\right)\ket{sm}\\
                            &= m \ket{sm}&
                            &= s(s + 1)\ket{sm}.
    \end{align*}
    Desse modo, concluímos que os estados considerados são autoestados simultâneos de \(\vetor{S}^2\) e \(S_z.\)
\end{proof}

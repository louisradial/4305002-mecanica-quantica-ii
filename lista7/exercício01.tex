% vim: spl=pt
\begin{exercício}{Momento angular do campo eletromagnético}{ex1}
    Estabeleça a decomposição do momento angular do campo eletromagnético nas partes intrínseca e orbital, \(\vetor{J} = \vetor{S} + \vetor{L},\) onde
    \begin{equation*}
        \vetor{S} = \frac1{2c} \int_{\mathbb{R}^3} \dln3x \left(\vetor{E}\times \vetor{A} - \vetor{A} \times \vetor{E}\right)
    \end{equation*}
    e
    \begin{equation*}
        \vetor{L} = \frac1{2c} \int_{\mathbb{R}^3} \dln3x \sum_{i = 1}^3 \left[E_i (\vetor{x} \times \nabla A_i) + (\vetor{x} \times \nabla A_i) E_i\right].
    \end{equation*}
\end{exercício}
\begin{proof}[Resolução]
    Consideramos o momento angular do campo eletromagnético clássico,
    \begin{equation*}
        \vetor{J}_\mathrm{cl} = \frac1c \int_{\mathbb{R}^3} \dln3x \vetor{x} \times (\vetor{E} \times \vetor{B}).
    \end{equation*}
    Notemos que 
    \begin{align*}
        \inner{\vetor{e}_i}{\vetor{x} \times (\vetor{E} \times \vetor{B})} 
        &= \epsilon_{ijk} x_j \epsilon_{k\ell m} E_\ell B_{m}\\
        % &= \epsilon_{ijk} \epsilon_{k\ell m} \epsilon_{m ab} x_j E_\ell \partial_a A_b\\
        % &= \epsilon_{ijk} (\delta_{k a} \delta_{\ell b} - \delta_{k b} \delta_{\ell a}) x_j E_\ell \partial_a A_b\\
        &= \epsilon_{ijk} x_j E_\ell (\partial_k A_\ell - \partial_\ell A_k)\\
        &= E_\ell (\vetor{x} \times \nabla)_i A_\ell - \epsilon_{ijk}E_\ell x_j \partial_\ell A_k\\
        % &= E_\ell (\vetor{x} \times \nabla)_i A_\ell - \epsilon_{ijk} E_\ell \partial_\ell (x_j A_k) + \epsilon_{ijk} E_j A_k\\
        &= E_\ell (\vetor{x} \times \nabla)_i A_\ell - \epsilon_{ijk} \partial_\ell (E_\ell x_j A_k) + \epsilon_{ijk} E_j A_k\\
        &= E_{\ell} (\vetor{x} \times \nabla)_i A_\ell - \partial_\ell (E_\ell \vetor{x} \times \vetor{A})_i + (\vetor{E}\times \vetor{A})_i,
    \end{align*}
    onde usamos que \(\partial_\ell E_\ell = 0.\) Assim, temos a decomposição
    \begin{align*}
        \vetor{J}_\mathrm{cl} &= \frac1c \int_{\mathbb{R}^3} \dln3x \left[E_\ell (\vetor{x} \times \nabla) A_\ell - \partial_\ell (E_\ell \vetor{x} \times \vetor{A}) + \vetor{E} \times \vetor{A}\right]\\
                              &= \frac1c \int_{\mathbb{R}^3} \dln3x E_\ell (\vetor{x} \times \nabla) A_\ell + \frac1c \int_{\mathbb{R}^3} \dln3x \vetor{E} \times \vetor{A}
    \end{align*}
    do momento angular clássico. Como o primeiro termo depende da escolha de origem, o associamos ao momento angular orbital
    \begin{equation*}
        \vetor{L} = \frac1{2c} \int_{\mathbb{R}^3} \dln3x \left\{E_\ell (\vetor{x} \times \nabla) A_\ell - \left[(\vetor{x} \times \nabla) A_\ell\right] E_\ell\right\},
    \end{equation*}
    enquanto que
    \begin{equation*}
        \vetor{S} = \frac1{2c} \int_{\mathbb{R}^3} \dln3x (\vetor{E} \times \vetor{A} - \vetor{A} \times \vetor{E})
    \end{equation*}
    é o operador de momento angular intrínseco do campo eletromagnético.
\end{proof}

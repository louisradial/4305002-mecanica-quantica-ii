% vim: spl=pt
\begin{exercício}{Radiação emitida no efeito Zeeman}{ex4}
    Considere o efeito Zeeman nas transições \(E1\) \((j_i, m_i) \to (j_f, m_f).\) Mostre que o campo elétrico da radiação emitida é dada por
    \begin{equation*}
        \vetor{E} \propto \sum_{\lambda = \pm 1} \sum_{\kappa = -1}^1 \conj{\vetor{e}}_{\vetor{k}\lambda} d^{(1)}_{-\lambda, \kappa}(\theta) \braket{j_i m_i 1 \kappa}{j_f m_f},
    \end{equation*}
    onde \(d^{(1)}\) é a matriz de rotação para \(j = 1,\) \(\theta\) é o ângulo entre o momento \(\hbar \vetor{k}\) do fóton e o campo magnético. Note que a polarização, que é elíptica no caso geral, é completamente determinada pela simetria de rotação. Considere o caso especial em que \(\theta = 0\) e mostre que apenas polarização circular é emitida.
\end{exercício}
\begin{proof}[Resolução]
    
\end{proof}

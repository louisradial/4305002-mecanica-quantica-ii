% vim: spl=pt
\begin{exercício}{Transições de radiação}{ex5}
    Frequentemente mais de um multipolo pode contribuir uma transição de radiação. Se \(J^P\) representa um estado de momento angular \(J\) e paridade \(P,\) mostre que na transição \(4^- \to 1^+\) as contribuições de \(E3,\) \(M4,\) e \(E5\) são permitidas e são adicionadas coerentemente à amplitude de transição. Mostre também que a contribuição dominante é \(E3,\) seguida pela interferência \(E3{-}M4.\)
\end{exercício}
\begin{proof}[Resolução]
    
\end{proof}

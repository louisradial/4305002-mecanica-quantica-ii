% vim: spl=pt
\documentclass[portuguese,biblatex]{artigo}
\usepackage{dirac}

\DeclareMathOperator{\texpl}{\mathcal{T}_{\leftarrow}\exp}
\DeclareMathOperator{\texpr}{\mathcal{T}_{\rightarrow}\exp}
\newcommand\reduced{\mathcal{S}}
\newcommand\bath{\mathcal{B}}

\addbibresource{bibliografia.bibtex}
\title{Sistemas quânticos abertos}
\author{Louis Bergamo Radial}

\begin{document}

\maketitle

\section{Evolução temporal de sistemas fechados}
\nocite{breuer}
O postulado de evolução unitária garante a existência de um operador unitário \(U(t,t')\) agindo no espaço de Hilbert \(\mathscr{H}\) que é associado ao sistema. A evolução temporal do estado \(\ket{\psi; t} \in \mathscr{H}\) é dada por
\begin{equation*}
    i\diffp*{\ket{\psi; t}}{t} = H(t) \ket{\psi; t},
\end{equation*}
portanto definindo \(U(t, t')\) por \(\ket{\psi;t} = U(t,t') \ket{\psi; t}\) obtemos
\begin{equation*}
    i \diffp{U(t, t')}{t} = H(t) U(t,t')
\end{equation*}
com \(U(t', t') = \unity\) para todo \(t'.\) Assim, podemos formalmente escrever o operador evolução com a série de Dyson\cite{barata}
\begin{equation*}
    U(t, t') = \texpl\left[-i \int_{t'}^{t} \dli{s} H(s)\right] = \unity + \sum_{m = 1}^\infty (-i)^m \int_{t'}^{t} \dli{s_1} \int_{t'}^{s_1} \dli{s_2} \dots \int_{t'}^{s_{m-1}} \dli{s_m} H(s_1) \dots H(s_m)
\end{equation*}
e o seu adjunto por
\begin{equation*}
    \herm{U}(t,t') = \texpr\left[i \int_{t'}^t \dli{s} H(s)\right] = \unity + \sum_{m = 1}^\infty i^m \int_{t'}^t \dli{s_1} \dots \int_{t'}^{s_{m-1}} \dli{s_m} H(s_m) \dots H(s_1),
\end{equation*}
onde \(\mathcal{T}_\leftarrow\) representa o ordenamento temporal crescente para a esquerda e analogamente para \(\mathcal{T}_\rightarrow\). No caso em que \(\commutator{H(s)}{H(t)} = 0\) para todos \(t\) e \(s\), a série de Dyson se simplifica para
\begin{equation*}
    U(t,t') = \exp\left[-i \int_{t'}^t \dli{s} H(s)\right],
\end{equation*}
e o caso particular de sistemas \emph{isolados}, em que \(H(t)\) é independente do tempo, para
\begin{equation*}
    U(t,t') = \exp\left[-i (t - t')H\right],
\end{equation*}
que é condizente com o teorema de Stone \cite{reedsimon1}.

\subsection{Matriz densidade}
Para estados descritos por uma matriz densidade \(\rho(t_0)\) no instante \(t_0\) dada por
\begin{equation*}
    \rho(t_0) = \sum_{\nu} p_\nu\ketbra{\nu}{\nu}
\end{equation*}
temos a evolução temporal
\begin{equation*}
    \rho(t) = \sum_\nu p_\nu U(t, t_0)\ketbra{\nu}{\nu}\herm{U}(t, t_0) = U(t,t_0) \rho(t_0) \herm{U}(t,t_0)
\end{equation*}
logo a matriz densidade satisfaz a equação de Liouville-von Neumann
\begin{equation*}
    i\diff{\rho(t)}{t} = \commutator{H(t)}{\rho(t)} = \mathcal{L}(t)\left\{\rho(t)\right\}
\end{equation*}
onde \(\mathcal{L}(t)\) é o operador linear agindo no espaço de operadores dado por \(\mathcal{L}(t)\left\{A\right\} = \commutator{H(t)}{A}\). Podemos escrever a solução formal
\begin{equation*}
    \rho(t) = \texpl\left[-i \int_{t_0}^t\dli{s} \mathcal{L}(t)\right]\left\{\rho(t_0)\right\},
\end{equation*}
e temos pelo lema de Campbell
\begin{equation*}
    e^{sX} Y e^{-sX} = Y + \sum_{m = 1}^\infty \frac{s^m\commutator{X}{Y}^{[m]}}{m!},
    \quad\text{com}\quad
    \commutator{X}{Y}^{[k+1]} = \commutator{X}{\commutator{X}{Y}^{[k]}}
    \quad\text{e}\quad
    \commutator{X}{Y}^{[1]} = \commutator{X}{Y},
\end{equation*}
que no caso de sistemas isolados vale
\begin{align*}
    \rho(t) &= \exp\left[-i (t-t_0) \mathcal{L}\right]\left\{\rho(t_0)\right\}\\
            &= \rho(t_0) + \sum_{m = 1}^\infty \frac{[-i(t - t_0)]^m}{m!} \commutator{H}{\rho(t_0)}^{[m]}\\
            &= e^{-i (t-t_0) H} \rho(t_0) e^{i (t - t_0) H}
            % &= U(t, t_0) \rho(t_0) \herm{U}(t, t_0)
\end{align*}
como esperado.

Um estado \(\rho\) é puro se tem um único autovalor igual a 1, ou, equivalentemente, se \(\Tr \rho^2 = 1.\) Outra formulação equivalente é se a entropia de von Neumann, \(S(\rho) = - \Tr(\rho \ln \rho),\) se anula, onde temos \(s(x) = - x \ln x\) para \(x \in (0,1]\) e \(s(0) = 0\). De fato, utilizando a decomposição espectral
\begin{equation*}
    \rho = \sum_i p_i \ketbra{i}{i},
\end{equation*}
onde \(p_i \in [0,1]\) pela positividade de \(\rho,\) temos
\begin{equation*}
    S(\rho) = -\sum_i p_i \ln p_i,
\end{equation*}
portanto \(S(\rho) = 0\) se e somente se \(p_i \in \set{0,1}\) para todos os estados da base. De \(\Tr\rho = 1,\) concluímos que \(S(\rho) = 0\) se e somente se \(\rho\) é puro. Como a entropia de von Neumann depende apenas do espectro do estado, a evolução unitária leva estados puros em estados puros.

\subsection{Representação de interação}
Para a representação de interação, escrevemos o hamiltoniano como \(H(t) = H_0(t) + V(t)\) e definimos os operadores
\begin{equation*}
    U_0(t, t') = \texpl\left[-i \int_{t'}^t \dli{s} H_0(s)\right]\quad\text{e}\quad
    U_I(t, t') = \herm{U}_0(t, t') U(t, t'),
\end{equation*}
onde \(U(t,t')\) é o operador de evolução temporal. O valor esperado de um operador \(A(t)\) é dado por
\begin{align*}
    \mean{A(t)} &= \Tr\left[A(t) \rho(t)\right]\\
                &= \Tr\left[A(t) U(t, t_0) \rho(t_0) \herm{U}(t, t_0)\right]\\
                &= \Tr\left[A(t) U_0(t, t_0) U_I(t, t_0) \rho(t_0) \herm{U}_I(t, t_0) \herm{U}_0(t, t_0)\right]\\
                &= \Tr\left[\herm{U}_0(t, t_0)A(t)U_0(t, t_0) U_I(t, t_0) \rho(t_0) \herm{U}_I(t, t_0)\right]
\end{align*}
então definimos os operadores na representação de interação por
\begin{equation*}
    A_I(t) = \herm{U}_0(t, t_0) A(t) U_0(t, t_0)
\end{equation*}
e a matriz densidade por
\begin{equation*}
    \rho_I(t) = U_I(t, t_0) \rho(t_0) \herm{U}_I(t, t_0),
\end{equation*}
de forma que
\begin{equation*}
    \Tr[A(t) \rho(t)] = \mean{A(t)} = \Tr[A_I(t) \rho_I(t)],
\end{equation*}
isto é, a representação não muda o conteúdo físico. Para obter a equação de evolução temporal para estados nessa representação, notemos que
\begin{align*}
    i\diffp{U_I(t, t')}{t} &= -\herm{U}_0(t, t') H_0(t) U(t, t') + \herm{U}_0(t, t') H(t) U(t, t')\\
                           &= \herm{U}_0(t, t') V(t) U(t, t')\\
                           &= \herm{U}_0(t, t') V(t) U_0(t,t') \herm{U}_0(t,t') U(t, t')\\
                           &= V_I(t) U_I(t,t'),
\end{align*}
portanto
\begin{equation*}
    U_I(t,t') = \texpl\left[-i \int_{t'}^t\dli{s} V_I(s)\right]
\end{equation*}
e então
\begin{equation*}
    i\diffp*{\ket{\psi; t}_I}{t} = V_I(t) \ket{\psi; t}_I,
\end{equation*}
onde \(\ket{\psi; t_0} = \ket{\psi; t_0}_I\) e \(\ket{\psi; t}_I = U_I(t, t_0) \ket{\psi; t_0}_I.\)

Na representação de interação, os operadores satisfazem
\begin{align*}
    i \diffp{A_I(t)}{t} &= - \herm{U}_0(t,t_0) H_0(t)A(t) U_0(t,t_0) + \herm{U}_0(t,t_0) A(t) H_0(t) U_0(t, t_0) + i\herm{U}_0(t,t_0) \diffp{A(t)}{t} U_0(t,t_0)\\
                        &= \herm{U}_0(t,t_0) \commutator{A(t)}{H_0(t)} U_0(t,t_0) + i \herm{U}_0(t,t_0) \diffp{A(t)}{t} U_0(t,t_0)\\
                        &= \commutator{A_I(t)}{H_0(t)} + i \left(\diffp{A(t)}{t}\right)_I.
\end{align*}
Notemos que no caso \(H_0(t) = 0,\) recuperamos a representação de Schrödinger, enquanto que para \(V(t) = 0\) recuperamos a representação de Heisenberg. Nesta última apenas os operadores evoluem temporalmente, e temos
\begin{align*}
    \diff{\mean{A(t)}}{t} &= \diff*{\Tr[\rho(t) A(t))]}{t}\\
                          &= \Tr\left[\rho_H(t_0)\diff*{A_H(t)}{t}\right]\\
                          &= \Tr\left\{\rho_H(t_0)\left[i \commutator{H_H(t)}{A_H(t)} + \left(\diffp{A(t)}{t}\right)_H\right]\right\}\\
                          &= \mean*{i\commutator{H(t)}{A(t)}} + \mean*{\diffp{A(t)}{t}},
\end{align*}
resultado do teorema de Ehrenfest.

\section{Evolução temporal de sistemas abertos}
\nocite{susana}
Vamos considerar agora sistemas compostos por um sistema de interesse \(\reduced\), ou sistema \emph{reduzido}, e um sistema ambiente \(\bath,\) de forma que o espaço de Hilbert do sistema total é \(\mathscr{H} = \mathscr{H}_\reduced \otimes \mathscr{H}_\bath.\) Os observáveis de interesse são da forma \(A \otimes \unity_\bath,\) portanto a descrição de sistemas abertos se baseia na matriz densidade reduzida \(\rho_\reduced = \Tr_\bath\rho,\) já que medidas de tais quantidades são dadas por
\begin{equation*}
    \mean{A} = \Tr\left[\rho(A \otimes \unity_\bath) \right] = \Tr_\reduced(\rho_\reduced A).
\end{equation*}
Tomando o caso em que em algum instante \(t_0\) o estado do sistema total \(\rho(t_0)\) é não correlacionado e o estado do sistema reduzido é \(\rho_\reduced(t_0),\) então podemos escrever \(\rho(t_0) = \rho_\reduced(t_0) \otimes \rho_\bath(t_0).\) Da evolução temporal, temos \(\rho(t) = U(t, t_0) [\rho_\reduced(t_0) \otimes \rho_\bath(t_0)] \herm{U}(t, t_0)\) e então o estado do sistema reduzido é dado por
\begin{equation*}
    \rho_\reduced(t) = \Tr_\bath\left\{U(t,t_0) [\rho_\reduced(t_0) \otimes \rho_\bath(t_0)] \herm{U}(t, t_0)\right\}.
\end{equation*}
No caso particular em que o operador evolução se fatora \(U(t,t_0) = U_\reduced(t,t_0) \otimes U_\bath(t,t_0),\) sabemos que
\begin{equation*}
    \rho_\reduced(t) = U_\reduced(t,t_0) \rho_\reduced(t_0) \herm{U}_\reduced(t, t_0) = \mathcal{V}(t,t_0)\rho_\reduced(t_0),
\end{equation*}
onde definimos o superoperador \(\mathcal{V}(t,t_0)\) pela conjugação por \(U_\reduced(t,t_0).\)


\subsection{Mapa dinâmico}
É desejável então definir um mapa dinâmico \(\mathcal{V}(t,t_0)\) de tal forma que 
\begin{equation*}
    \rho_\reduced(t) = \mathcal{V}(t,t_0) \rho_\reduced(t_0),
\end{equation*}
isto é, uma aplicação que codifica toda a dinâmica com o subsistema \(\bath\). Entretanto, no caso geral o mapa dinâmico não depende apenas do operador de evolução \(U(t,t_0)\) e de \(\bath,\) mas também depende do estado inicial de \(\reduced.\) Para ver isso, consideramos que no instante \(t_0\) os estados de \(\reduced\) e de \(\bath\) são \(\rho_\reduced(t_0)\) e \(\rho_\bath(t_0)\) mas que possivelmente há correlação
\begin{equation*}
    \rho_c(t_0) = \rho(t_0) - \rho_\reduced(t_0) \otimes \rho_\bath(t_0),
\end{equation*}
de forma que
\begin{equation*}
    \rho(t_0) = \rho_\reduced(t_0) \otimes \rho_\bath(t_0) + \rho_c(t_0)
    \quad\text{e}\quad
    \Tr_\reduced(\rho_c) = 0 = \Tr_\bath(\rho_c).
\end{equation*}
Utilizamos a decomposição espectral de \(\rho_\bath(t_0),\)
\begin{equation*}
    \rho_\bath(t_0) = \sum_{\beta} \lambda_\beta \ketbra{\beta}{\beta},
\end{equation*}
e definimos os operadores
\begin{equation*}
    K_{\alpha \beta}(t,t_0) = \sqrt{\lambda_\beta} \bra{\alpha} U(t,t_0)\ket{\beta} \quad\text{e}\quad \herm{K}_{\alpha \beta}(t,t_0) = \sqrt{\lambda_\beta}\bra{\beta} \herm{U}(t,t_0)\ket{\alpha},
\end{equation*}
e então a evolução do sistema reduzida é dada por
\begin{align*}
    \rho_\reduced(t) &= \Tr_\bath\left\{ U(t,t_0) \left[\rho_\reduced(t_0) \otimes \rho_\bath(t_0) + \rho_c(t_0)\right] \herm{U}(t,t_0)\right\}\\
                     &= \sum_{\alpha,\beta} \lambda_\beta \bra{\alpha} U(t, t_0) [\rho_\reduced(t_0) \otimes \ketbra{\beta}{\beta} + \rho_c(t_0)] \herm{U}(t, t_0) \ket{\alpha} + \Tr_\bath\left[U(t,t_0) \rho_c(t_0) \herm{U}(t,t_0)\right]\\
                     &= \sum_{\alpha,\beta} K_{\alpha \beta}(t,t_0) \rho_\reduced(t_0) \herm{K}_{\alpha \beta}(t,t_0) + \Tr_\bath\left[U(t,t_0) \rho_c(t_0) \herm{U}(t,t_0)\right]\\
                     &= \mathcal{V}(t,t_0) \rho_\reduced(t_0).
\end{align*}
O primeiro termo do mapa dinâmico \(\mathcal{V}(t,t_0)\) depende apenas do operador de evolução e do estado de \(\bath,\) como desejado, enquanto que o outro termo depende de \(\rho_c(t_0),\) e, portanto, do estado inicial \(\rho_\reduced(t_0).\) 

\begin{example}{Evolução temporal de um qubit com ambiente de um qubit\cite{piotr}}{qubit}
    Considere um sistema de dois qubits com hamiltoniana \(H = \frac12\omega \sigma_z \otimes \sigma_z\) e estado inicial \(\rho(0) = X \otimes X,\) onde \(X = \ketbra{\rightarrow}{\rightarrow}\) e \(\sigma_x\ket{\rightarrow} = \ket{\rightarrow}.\) Determine a evolução temporal do sistema reduzido \(\rho_\reduced(t) = \Tr_\bath\rho(t)\) assim como a entropia de von Neumann, \(S(t) = -\Tr_\reduced\left[\rho_\reduced(t) \ln \rho_\reduced(t)\right].\)
\end{example}
\begin{proof}[Resolução]
    Como o estado \(\bath\) já está dado na sua decomposição espectral, devemos calcular apenas \(K_{\rightarrow \rightarrow}\) e \(K_{\leftarrow \rightarrow}\). Temos
    \begin{align*}
        K_{\rightarrow \rightarrow} &= \bra{\rightarrow}_\bath e^{-i \frac12 \omega t \sigma_z \otimes \sigma_z}  \ket{\rightarrow}_\bath&
        K_{\leftarrow \rightarrow} &= \bra{\leftarrow}_\bath e^{-i \frac12 \omega t \sigma_z \otimes \sigma_z}  \ket{\rightarrow}_\bath\\
                                      &= \sum_{m=0}^\infty \frac{\left(-\frac12i \omega t\right)^m}{m!} \sigma_z^m \bra{\rightarrow}\sigma_z ^m \ket{\rightarrow}&
                                      &= \sum_{m=0}^\infty \frac{\left(-\frac12i \omega t\right)^m}{m!} \sigma_z^m \bra{\leftarrow}\sigma_z ^m \ket{\rightarrow}\\
                                      &= \sum_{m=0}^\infty \frac{\left(-\frac12i \omega t\right)^m}{m!} \sigma_z^m \frac{\braket{\rightarrow}{\uparrow} + (-1)^m\braket{\rightarrow}{\downarrow}}{\sqrt{2}}&
                                      &= \sum_{m=0}^\infty \frac{\left(-\frac12i \omega t\right)^m}{m!} \sigma_z^m \frac{\braket{\leftarrow}{\uparrow} + (-1)^m\braket{\leftarrow}{\downarrow}}{\sqrt{2}}\\
                                      &= \frac12 e^{-\frac12 i \omega t \sigma_z} + \frac12 e^{i \frac12 i \omega t \sigma_z}&
                                      &= \frac12 e^{-\frac12 i \omega t \sigma_z} - \frac12 e^{i \frac12 i \omega t \sigma_z}\\
                                      &= \cos\left(\frac12 \omega t \sigma_z\right)&
                                      &= -i \sin\left(\frac12 \omega t \sigma_z\right)
    \end{align*}
    e então
    \begin{equation*}
        K_{\rightarrow\rightarrow} \ket{\rightarrow} = \cos\left(\frac12 \omega t\right)\ket{\rightarrow},
        \quad\text{e}\quad
        K_{\leftarrow\rightarrow} \ket{\rightarrow} = -i\sin\left(\frac12 \omega t\right) \ket{\leftarrow}.
    \end{equation*}
    Dessa forma, temos
    \begin{align*}
        \rho_\reduced(t) &= K_{\rightarrow\rightarrow}(t) \rho_\reduced(0) \herm{K}_{\rightarrow\rightarrow}(t) + K_{\leftarrow\rightarrow}(t) \rho_\reduced(0) \herm{K}_{\leftarrow\rightarrow}(t)\\
                         &= \ketbra{\rightarrow}{\rightarrow} \cos^2\left(\frac12 \omega t\right) + \ketbra{\leftarrow}{\leftarrow} \sin^2\left(\frac12 \omega t\right)\\
                         &= \sin^2\left(\frac12 \omega t\right) \unity + \cos(\omega t) \ketbra{\rightarrow}{\rightarrow}\\
                         &= \frac{1 - \cos(\omega t)}{2} \unity + \cos(\omega t) \rho_\reduced(0)\\
                         &= \frac12 \begin{pmatrix}
                             1 && \cos(\omega t)\\
                             \cos(\omega t) && 1
                         \end{pmatrix}
    \end{align*}
    como o estado do sistema reduzido no instante \(t\).

    Mostramos que
    \begin{equation*}
        \rho_\reduced(t) = \cos^2\left(\frac12 \omega t\right) \ketbra{\rightarrow}{\rightarrow} + \sin^2\left(\frac12 \omega t\right) \ketbra{\leftarrow}{\leftarrow},
    \end{equation*}
    portanto os autovalores de \(\rho_\reduced(t)\) são \(\set{\cos^2\left(\frac12 \omega t\right), \sin^2\left(\frac12 \omega t\right)}.\) Assim, 
    \begin{align*}
        S(t) &= -\Tr_\reduced\left[\rho_\reduced(t) \ln \rho_\reduced(t)\right]\\
             &= -\Tr_\reduced\left[\left(\cos^2\frac{\omega t}{2} \ketbra{\rightarrow}{\rightarrow} + \sin^2\frac{\omega t}{2} \ketbra{\leftarrow}{\leftarrow}\right] \left[\ln\cos^2\frac{\omega t}{2} \ketbra{\rightarrow}{\rightarrow} + \ln\sin^2\frac{\omega t}{2} \ketbra{\leftarrow}{\leftarrow}\right) \right]\\
             &= - \cos^2\frac{\omega t}{2} \ln \cos^2\frac{\omega t}{2} - \sin^2 \frac{\omega t}{2} \ln \sin^2\frac{\omega t}{2}
    \end{align*}
    é a entropia de von Neumann.
\end{proof}
O \cref{exam:qubit} mostra que a evolução temporal de um sistema aberto não é unitária em geral. De fato, para aquele sistema temos
\begin{equation*}
    S\left(\frac{n\pi}{\omega}\right) = 0\quad\text{e}\quad
    S\left(\frac{(n + \frac12)\pi}{\omega}\right) = \ln2,
\end{equation*}
isto é, o sistema reduzido oscila entre um estado puro e um estado misto.

\subsection{Mapas dinâmicos universais}
Vamos doravante supor que existe um instante \(t_0\) em que não há correlação. A partir disso definimos \emph{mapas dinâmicos universais} que agem como mapas dinâmicos, mas que, pelo cômputo anterior, dependem apenas da evolução do sistema composto e do estado do sistema ambiente.
\begin{equation*}
    \begin{tikzcd}[column sep = large, row sep = large]
        \rho(t_0) = \rho_\reduced(t_0) \otimes \rho_B(t_0) \arrow{rr}{\text{evolução unitária}} \arrow{d}{\Tr_\bath} && \rho(t) = U(t,t_0) \rho(t_0) \herm{U}(t,t_0)\arrow{d}{\Tr_\bath}\\
        \rho_\reduced(t_0) \arrow{rr}{\text{mapa dinâmico universal}} && \rho_\reduced(t) = \mathcal{V}(t,t_0)\rho_\reduced(t_0)
    \end{tikzcd}
\end{equation*}
\cite{susana} mostra que uma condição necessária e suficiente para que um mapa dinâmico seja universal é de que o estado \(\rho_\reduced(t_0)\) seja induzido por um estado produto \(\rho(t_0) = \rho_\reduced(t_0) \otimes \rho_\bath(t_0)\) com \(\rho_\bath(t_0)\) fixo para qualquer \(\rho_\reduced(t_0).\) 

Como feito acima,
\begin{equation*}
    \rho_\reduced(t) = \mathcal{V}(t,t_0) \rho_\reduced(t_0) = \sum_{\alpha, \beta} K_{\alpha \beta} (t,t_0) \rho_\reduced(t_0) \herm{K}_{\alpha \beta}(t,t_0)
\end{equation*}
é a evolução temporal do estado reduzido. Vamos mostrar um mapa dinâmico universal preserva o traço e a positividade do estado, apesar de não ser, em geral, uma transformação unitária. Para o traço, notemos que
\begin{align*}
    \sum_{\alpha\beta} \herm{K}_{\alpha\beta} K_{\alpha\beta} &= \sum_{\alpha\beta} \lambda_\beta \bra{\beta}  \herm{U}(t,t_0) \ket{\alpha} \bra{\alpha} U(t,t_0)\ket{\beta}\\
                                                              &= \sum_{\beta} \lambda_\beta \bra{\beta} \herm{U}(t,t_0) U(t,t_0) \ket{\beta}\\
                                                              &= \sum_\beta \lambda_\beta
                                                              = \Tr_\bath \rho_\bath
                                                              = 1,
\end{align*}
então
\begin{align*}
    \Tr_\reduced \rho_\reduced(t) &= \Tr_\reduced \left(\sum_{\alpha\beta} K_{\alpha\beta}(t) \rho_\reduced(t_0)\herm{K}_{\alpha\beta}(t)\right)\\
                                  &= \Tr_\reduced \left(\rho_\reduced(t_0) \sum_{\alpha\beta} \herm{K}_{\alpha\beta}(t) K_{\alpha\beta}(t)\right)\\
                                  &= \Tr_\reduced \rho_\reduced(t_0) = 1.
\end{align*}
Para a positividade, consideramos um vetor \(\ket{\psi} \in \mathscr{H}_\reduced\) e definimos 
\begin{equation*}
    \ket{\phi_{\alpha\beta};t} = \herm{K}_{\alpha\beta}(t)\ket{\psi},
\end{equation*}
então
\begin{equation*}
    \bra{\psi} \rho_\reduced(t)\ket{\psi} = \sum_{\alpha\beta} \bra{\psi} K_{\alpha\beta}(t) \rho_\reduced(t_0) \herm{K}_{\alpha\beta}(t)\ket{\psi}
                                          = \sum_{\alpha\beta} \bra{\phi_{\alpha\beta}; t} \rho_\reduced(t_0) \ket{\phi_{\alpha\beta}; t}
                                          \geq 0,
\end{equation*}
já que \(\rho_\reduced(t_0)\) é positiva.

É importante notar que, ao contrário da evolução unitária, não temos, em geral, uma lei de composição como
\begin{equation*}
    U(t,t_0) = U(t,t') U(t', t_0)
\end{equation*}
para mapas dinâmicos universais. De fato, o estado do sistema no instante \(t'\) pode ter correlação, e então um mapa dinâmico \(\mathcal{V}(t, t')\) não seria universal. De toda forma, a dinâmica do sistema reduzido é dada pela família de superoperadores \(\family{\mathcal{V}(t,t_0)}{t \geq t_0},\) com \(\mathcal{V}(t_0,t_0) = \unity.\)



\section{Equação de Lindblad}
Partindo da equação de Liouville-von Neumann para o estado do sistema composto,
\begin{equation*}
    i\diff{\rho(t)}{t} = \commutator{H(t)}{\rho(t)}
\end{equation*}
podemos tomar o traço parcial e obter
\begin{equation*}
    i\diff{\rho_\reduced(t)}{t} = \Tr_\bath\commutator{H(t)}{\rho(t)}.
\end{equation*}
Gostaríamos de expressar a equação acima apenas em termos de operadores que agem no sistema \(\reduced,\) já que a equação acima pode ser tecnicamente difícil de se resolver e, mais importante, podemos não saber a dinâmica do sistema ambiente. Assim, desejamos encontrar uma equação na forma de Lindblad,
\begin{equation*}
    \diff{\rho_\reduced(t)}{t} = \mathcal{L}\rho_\reduced(t),
\end{equation*}
onde o superoperador \(\mathcal{L},\) chamado de Lindbladiano do sistema reduzido, atua apenas no espaço das matrizes densidades de \(\reduced.\) O Lindbladiano seria então, se existir, um gerador da evolução do sistema, 
\begin{equation*}
    \mathcal{V}(t,t_0) = \texpl\left[\int_{t_0}^{t} \dli{s} \mathcal{L}(s)\right].
\end{equation*}

A fim de evitar tecnicalidades, vamos tomar o caso de um espaço de Hilbert de dimensão finita para o sistema reduzido, \(\dim\mathscr{H}_\reduced = N \in \mathbb{N}.\) Seja então \(\set{F_i}{1 \leq N^2}\) uma base ortonormal de operadores agindo em \(\mathscr{H}_\reduced,\) com
\begin{equation*}
    \inner{F_i}{F_j} = \Tr_\reduced \herm{F}_iF_j = \delta_{ij}
\end{equation*}
e com a escolha de traço
\begin{equation*}
    F_{N^2} = \frac{1}{\sqrt{N}} \unity_\reduced\quad\text{e}\quad \Tr F_{i} = 0, \forall i < N^2.
\end{equation*}
Nesse caso os operadores da decomposição do mapa dinâmico são dados por
\begin{equation*}
    K_{\alpha\beta}(t) = \sum_{i = 1}^{N^2} \inner{F_i}{K_{\alpha \beta}(t)}F_i
\end{equation*}
e obtemos
\begin{align*}
    \mathcal{V}(t,t_0) \rho_\reduced(t_0) &= \sum_{\alpha\beta}{\sum_{i = 1}^{N^2} \sum_{j = 1}^{N^2} \inner{F_i}{K_{\alpha \beta}(t)} \inner{F_j}{K_{\alpha\beta}(t)}^* F_i \rho_\reduced(t_0) \herm{F}_j}\\
                                          &= \sum_{i = 1}^{N^2} \sum_{j= 1}^{N^2} c_{ij}(t) F_i \rho_\reduced(t_0) \herm{F}_j,
\end{align*}
onde definimos os coeficientes
\begin{equation*}
    c_{ij} = \sum_{\alpha\beta}{\inner{F_i}{K_{\alpha\beta}(t)} \inner{F_j}{K_{\alpha\beta}(t)}^*}.
\end{equation*}
É fácil ver que a matriz dos coeficientes \(c_{ij}\) é hermitiana,
\begin{equation*}
    c_{ji}^* = \sum_{\alpha\beta}{\inner{F_j}{K_{\alpha\beta}(t)}^* \inner{F_i}{K_{\alpha\beta}(t)}}= \sum_{\alpha\beta}{\inner{F_i}{K_{\alpha\beta}(t)} \inner{F_j}{K_{\alpha\beta}(t)}^*}= c_{ij}
\end{equation*}
e positiva,
\begin{align*}
    \sum_{i=1}^{N^2} \sum_{j=1}^{N^2} v_i^* c_{ij}  v_j &= \sum_{i=1}^{N^2} \sum_{j=1}^{N^2}{v_i^* v_j \sum_{\alpha\beta}{\inner{F_i}{K_{\alpha\beta}(t)} \inner{F_j}{K_{\alpha\beta}(t)}^*}}\\
                                &= \sum_{\alpha\beta}{\inner*{\sum_{i = 1}^{N^2} v_i F_i}{K_{\alpha\beta}(t)} \inner*{\sum_{j = 1}^{N^2} v_j F_j}{K_{\alpha\beta}(t)}^*}\\
                                &= \sum_{\alpha\beta}{\abs*{\inner*{\sum_{i = 1}^{N^2} v_i F_i}{K_{\alpha\beta}(t)}}^2}\\
                                &\geq 0.
\end{align*}

Para que haja um gerador\cite{reedsimon1}, vamos supor que a família \(\family{\mathcal{V}(t,t_0)}{t \geq t_0}\) é contínua e então 
\begin{equation*}
    \mathcal{L} \rho_\reduced(t_0) = \lim_{\varepsilon \to 0} \frac{1}{\varepsilon} \left[\mathcal{V}(t_0 + \varepsilon, t_0) \rho_\reduced(t_0) - \rho_\reduced(t_0)\right]
                                   = \lim_{\varepsilon \to 0} \frac{1}{\varepsilon} \left\{\sum_{i = 1}^{N^2} \sum_{j = 1}^{N^2} \left[c_{ij}(t_0 + \varepsilon) - c_{ij}(t_0)\right] F_i \rho_\reduced(t_0) \herm{F}_j\right\},
\end{equation*}
onde usamos que \(\mathcal{V}(t_0,t_0) = \unity.\) Além disso, devemos ter \(c_{ij}(t_0) = N \delta_{i N^2} \delta_{j N^2}.\) Vamos brevemente limpar a notação e escrever \(t_0 = 0,\) \(\rho_\reduced(t_0) = \rho_\reduced,\) então da afirmação anterior temos
\begin{align*}
    \mathcal{L}\rho_\reduced &= \lim_{\varepsilon \to 0} \left\{\frac{c_{N^2 N^2}(\varepsilon) - N}{N \varepsilon}\rho_\reduced + \sum_{i = 1}^{N^2-1}\left[\frac{c_{iN^2}(\varepsilon)}{\sqrt{N} \varepsilon} F_i \rho_\reduced+\frac{c_{N^2i}(\varepsilon)}{\sqrt{N} \varepsilon} \rho_\reduced \herm{F}_i\right] + \sum_{i = 1}^{N^2-1}\sum_{j = 1}^{N^2 - 1} \frac{c_{ij}(\varepsilon)}{\varepsilon}F_i \rho_\reduced \herm{F}_j\right\},
\end{align*}
portanto com as definições
\begin{equation*}
    a_{N^2 N^2} = \lim_{\varepsilon \to 0} \frac{c_{N^2 N^2}(t_0 + \varepsilon) - N}{\varepsilon},\quad
    a_{i N^2} = \lim_{\varepsilon \to 0} \frac{c_{i N^2}(t_0 + \varepsilon)}{\varepsilon},\quad\text{e}\quad
    a_{ij} = \lim_{\varepsilon \to 0}\frac{c_{ij}(t_0 + \varepsilon)}{\varepsilon},
\end{equation*}
onde \(i,j \in \set{1, \dots, N^2 - 1},\) temos
\begin{equation*}
    \mathcal{L}\rho_\reduced = \frac{a_{N^2N^2}}{N} \rho_\reduced + \frac{1}{\sqrt{N}}\sum_{i = 1}^{N^2 - 1} \left(a_{iN^2} F_i \rho_\reduced + a_{iN^2}^* \rho_\reduced \herm{F}_i\right) + \sum_{i=1}^{N^2 - 1} \sum_{j=1}^{N^2 - 1} a_{ij} F_i \rho_\reduced \herm{F}_j.
\end{equation*}
Para escrever o Lindbladiano em uma forma familiar ao operador de Liouville-von Neumann, definimos ainda
\begin{equation*}
    F = \frac{1}{\sqrt{N}} \sum_{i = 1}^{N^2 -1} a_{i N^2} F_i,\quad
    G = \frac1{2N} a_{N^2 N^2}  + \frac12 (F + \herm{F}),\quad\text{e}\quad
    H = \frac{1}{2i}(\herm{F} - F),
\end{equation*}
e temos
\begin{align*}
    \mathcal{L}(t_0)\rho_S(t_0) &= \frac{a_{N^2 N^2}}{N}\rho_\reduced + F \rho_\reduced + \rho_\reduced \herm{F} +  \sum_{i = 1}^{N^2-1} \sum_{j = 1}^{N^2 - 1} a_{ij} F_i \rho_\reduced \herm{F}_j\\
                      &= \frac{a_{N^2 N^2}}{N}\rho_\reduced + \frac{F + \herm{F}}{2} \rho_\reduced + \frac{F - \herm{F}}{2}\rho_\reduced + \rho_\reduced \herm{F} +  \sum_{i = 1}^{N^2-1} \sum_{j = 1}^{N^2 - 1} a_{ij} F_i \rho_\reduced \herm{F}_j\\
                      &= G \rho_\reduced + \frac{a_{N^2 N^2}}{2N} - i H \rho_\reduced + \rho_\reduced \herm{F} + \sum_{i = 1}^{N^2- 1} \sum_{j = 1}^{N^2 - 1} a_{ij} F_i \rho_\reduced \herm{F}_j\\
                      &= G\rho_\reduced + \rho_\reduced G + i \rho_\reduced H - i H \rho_\reduced + + \sum_{i = 1}^{N^2- 1} \sum_{j = 1}^{N^2 - 1} a_{ij} F_i \rho_\reduced \herm{F}_j\\
                      &= \anticommutator{G(t_0)}{\rho_\reduced(t_0)} - i \commutator{H(t_0)}{\rho_\reduced(t_0)} + \sum_{i = 1}^{N^2-1} \sum_{j = 1}^{N^2 -1} a_{ij}(t_0) F_i \rho_\reduced(t_0) \herm{F}_j.
\end{align*}
\todo[Como o mapa dinâmico preserva o traço, temos]
\begin{equation*}
    \Tr_\reduced \mathcal{L}\rho_\reduced(t) = \Tr_\reduced \diff{\rho_\reduced(t)}{t} = \diff*{\Tr_\reduced \rho_\reduced(t)}{t} = 0,
\end{equation*}
e então
\begin{align*}
    0 &= \Tr_\reduced \mathcal{L}\rho_\reduced(t)\\
      &= \Tr_\reduced\left(\anticommutator{G}{\rho_\reduced} - i \commutator{H}{\rho_\reduced} + \sum_{i = 1}^{N^2-1} \sum_{j = 1}^{N^2 -1} a_{ij} F_i \rho_\reduced \herm{F}_j\right)\\
      &= \Tr_\reduced\left[\left(2G + \sum_{i = 1}^{N^2-1}\sum_{j =1}^{N^2-1} a_{ij} \herm{F}_j F_i\right)\rho_\reduced\right]
\end{align*}
e podemos concluir que
\begin{equation*}
    G = -\frac12 \sum_{i,j}^{N^2 - 1} a_{ij} \herm{F}_j F_i,
\end{equation*}
já que a equação acima é válida para qualquer \(\rho_\reduced.\) Uma outra simplificação que podemos fazer é usar a positividade de \(c_{ij}\) para diagonalizar \(a_{ij}\) com 
\begin{equation*}
    \sum_{i = 1}^{N^2-1} \sum_{j=1}^{N^2 -1}u_{ki} a_{ij} u^*_{\ell j} = \gamma_k \delta_{k \ell},\quad\text{com}\quad \gamma_k \geq 0,
\end{equation*}
e definir os operadores
\begin{equation*}
    A_j = \sum_{k = 1}^{N^2 - 1} u^*_{ji} F_i \iff F_i = \sum_{k = 1}^{N^2-1} u_{ki} A_k,
\end{equation*}
de forma que
\begin{equation*}
    \sum_{i = 1}^{N^2-1} \sum_{j=1}^{N^2 -1} a_{ij} F_i \rho_\reduced \herm{F}_j = \sum_{i = 1}^{N^2 - 1}\sum_{j = 1}^{N^2 - 1}\sum_{k = 1}^{N^2 - 1}\sum_{\ell = 1}^{N^2 - 1} u_{ki} a_{ij} u^*_{\ell j} A_k \rho_\reduced \herm{A}_\ell = \sum_{k = 1}^{N^2 - 1} \gamma_k A_k \rho_\reduced \herm{A}_k
\end{equation*}
e
\begin{equation*}
    \sum_{i = 1}^{N^2 - 1} \sum_{j = 1}^{N^2 - 1} a_{ij} \anticommutator{\herm{F}_j F_i}{\rho_\reduced(t_0)} = \sum_{k = 1}^{N^2 - 1} \gamma_k \anticommutator{\herm{A}_k A_k}{\rho_\reduced}
\end{equation*}
analogamente. Assim, o Lindbladiano é dado por
\begin{align*}
    \mathcal{L}(t_0) \rho_\reduced(t_0) &= - i \commutator*{H(t_0)}{\rho_\reduced(t_0)} + \sum_{i=1}^{N^2 - 1} \sum_{j = 1}^{N^2-1} a_{ij}(t_0) \left[F_i \rho_\reduced(t_0) \herm{F}_j - \frac12 \anticommutator*{\herm{F}_j F_i}{\rho_\reduced(t_0)}\right]\\
                                        &= -i \commutator*{H(t_0)}{\rho_\reduced(t_0)} + \sum_{k = 1}^{N^2 -1} \gamma_k \left(A_k \rho_\reduced \herm{A}_k - \frac12 \anticommutator*{\herm{A}_k A_k}{\rho_\reduced(t_0)}\right)
                                        &
\end{align*}
e podemos escrever a equação de Lindblad
\begin{equation*}
    \diff{\rho_\reduced(t)}{t} = -i \commutator*{H(t_0)}{\rho_\reduced(t_0)} + \mathcal{D}(t_0) \rho_\reduced(t_0),
\end{equation*}
onde definimos o \emph{dissipador} por
\begin{equation*}
    \mathcal{D}(t_0)\rho_\reduced(t_0) = \sum_{k = 1}^{N^2 -1} \gamma_k \left(A_k \rho_\reduced \herm{A}_k - \frac12 \anticommutator*{\herm{A}_k A_k}{\rho_\reduced(t_0)}\right).
\end{equation*}


\printbibliography
\end{document}

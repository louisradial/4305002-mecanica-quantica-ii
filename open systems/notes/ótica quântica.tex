% vim: spl=pt
\chapter{Ótica quântica}
Vamos restringir a discussão a um sistema de ótica quântica, em que o sistema de interesse \(\reduced\) é um sistema ligado que interage com o campo de radiação, que configura o ambiente \(\bath\). O Hamiltoniano livre do campo é dado por
\begin{equation}
    H_\bath = \sum_{\vetor{k}} \sum_{\lambda = 1}^2 \omega_k \mode{\herm{b}}{k}{\lambda} \mode{b}{k}{\lambda},
\end{equation}
decompondo o campo nos modos naturais em um volume \(\Omega\) com condições de contorno periódicas, relembrando que os modos são dados pelo momento \(\vetor{k}\) e polarização \(\lambda\) satisfazendo
\begin{equation}
    \vetor{k} \cdot \mode{\vetor{e}}{k}{\lambda} = 0,\quad
    \mode{\vetor{e}}{k}{\lambda} \cdot \mode{\vetor{e}}{k}{\lambda'} = \delta_{\lambda \lambda'}\quad\text{e}\quad
    \sum_{\lambda = 1}^2 \mode{e^i}{k}{\lambda} \mode{e^j}{k}{\lambda} = \delta_{ij} - \frac{k^ik^j}{\norm{\vetor{k}}^2} = \deltatr{k}{ij}
\end{equation}
e a relação de dispersão é dada por \(\omega_k = \norm{\vetor{k}}\) nas unidades naturais. O termo de interação considerado é dado pela aproximação de dipolo,
\begin{equation}
    V = - \vetor{D} \cdot \vetor{E},
\end{equation}
onde \(\vetor{D}\) é o operador de dipolo de \(\reduced\) e o campo elétrico \(\vetor{E}\) é dado por
\begin{equation}
    \vetor{E} = i \sum_{\vetor{k}} \sum_{\lambda = 1}^2 \sqrt{\frac{2\pi \omega_k}{\Omega}} \mode{\vetor{e}}{k}{\lambda} \left(\mode{b}{k}{\lambda} - \mode{\herm{b}}{k}{\lambda}\right)
\end{equation}
na representação de Schrödinger e por
\begin{equation}
    \vetor{E}(t) = i \sum_{\vetor{k}} \sum_{\lambda = 1}^2 \sqrt{\frac{2\pi \omega_k}{\Omega}} \mode{\vetor{e}}{k}{\lambda} \left(e^{-i \omega_k t} \mode{b}{k}{\lambda} - e^{i \omega_k t} \mode{\herm{b}}{k}{\lambda}\right)
\end{equation}
na representação de interação.

Como antes, definimos os operadores
\begin{equation}
    \vetor{A}(\omega) = \sum_{\epsilon' - \epsilon = \omega} \Pi(\epsilon) \vetor{D} \Pi(\epsilon')
\end{equation}
que decompõem o operador de dipolo na representação de interação,
\begin{equation}
    \vetor{D}(t) = \sum_{\omega} e^{-i\omega t} \vetor{A}(\omega) = \sum_{\omega} e^{i \omega t} \herm{\vetor{A}}(\omega),
\end{equation}
e podemos escrever o Hamiltoniano de interação
\begin{equation}
    V_I(t) = - \sum_{\omega} e^{-i \omega t} \vetor{A}(\omega) \cdot \vetor{E}(t).
\end{equation}
Assumindo que \(\mean{\vetor{E}(t)} = \vetor{0}\) na média do ambiente, segue da aproximação de Born-Markov que
\begin{equation}
    \diff*{\rho(t)}{t} = \sum_{\omega, \omega'} \sum_{i,j = 1}^3 e^{i(\omega - \omega')t} \Gamma_{ij}(t, \omega')\commutator{A_i(\omega') \rho(t)}{\herm{A}_j(\omega)} + \mathrm{h.c.}
\end{equation}
é a equação mestre para o estado reduzido na representação de interação, onde a matriz \(\Gamma_{ij}(t, \eta)\) é definida por
\begin{equation}
    \Gamma_{ij}(t, \eta) = \int_0^\infty \dli{s} e^{i \eta s} \mean{E_i(t) E_j(t - s)}
\end{equation}
e recebe o nome de tensor de correlação espectral.

Podemos escrever o tensor de correlação espectral como função dos valores esperados de produtos de operadores de criação e de aniquilação. Para limpar a notação, escrevemos
\begin{equation}
    \int_0^\infty \dli{s} \sum_{\vetor{k}, \lambda}\sum_{\vetor{k}', \lambda'} \frac{2\pi \sqrt{\omega_k \omega_{k'}}}{L} \mode{e^i}{k}{\lambda} \mode{e^j}{k'}{\lambda'} \to \sumint \dli{s} c_{ij}
\end{equation}
e então
\begin{align}
    \Gamma_{ij}(t, \eta) &= -\sumint \dli{s} c_{ij} e^{i \eta s} \mean*{\left(e^{-i \omega_k t} \mode{b}{k}{\lambda} - e^{i \omega_k t} \mode{\herm{b}}{k}{\lambda}\right)\left(e^{-i \omega_{k'} (t - s)} \mode{b}{k'}{\lambda'} - e^{i \omega_{k'} (t - s)} \mode{\herm{b}}{k'}{\lambda'}\right)}\\
                         &= \sumint \dli{s} c_{ij} \left(
                             \mean{\mode{b}{k}{\lambda} \mode{\herm{b}}{k'}{\lambda'}} e^{-i (\omega_k - \omega_{k'}) t - i (\omega_{k'} - \eta) s}
                             -\mean{\mode{b}{k}{\lambda} \mode{b}{k'}{\lambda'}} e^{-i (\omega_k + \omega_{k'}) t + i (\omega_{k'} + \eta) s}\right. + \nonumber\\
                         &{}\phantom{=\sumint \dli{s} c_{ij} }  \left. {}+ 
                             \mean{\mode{\herm{b}}{k}{\lambda}\mode{b}{k'}{\lambda'}} e^{-i(\omega_{k'} - \omega_k)t + i(\omega_{k'} + \eta)s} -
                             \mean{\mode{\herm{b}}{k}{\lambda}\mode{\herm{b}}{k'}{\lambda'}} e^{i(\omega_{k'} + \omega_k)t - i(\omega_{k'} - \eta)s}
                             \right).
\end{align}

\section{Banho térmico de radiação}
Vamos tomar o caso em que o campo eletromagnético está em um estado de temperatura \(T,\)
\begin{equation}
    \rho_\bath = \frac{1}{Z} \exp(-\beta H_\bath),
\end{equation}
onde a função de partição é dada por
\begin{equation}
    \ln Z = - \sum_{\vetor{k}}\sum_{\lambda} \ln\left(1 - e^{-\beta \omega_k}\right).
\end{equation}
Neste caso, segue de
\begin{equation}
    e^{\beta H_\bath} \mode{b}{k}{\lambda} e^{- \beta H_\bath} = e^{- \beta \omega_k} \mode{b}{k}{\lambda}
\end{equation}
que 
\begin{equation}
    \mean{\mode{b}{k}{\lambda}} = \mean{\mode{\herm{b}}{k}{\lambda}} = \mean{\mode{b}{k}{\lambda}\mode{b}{k'}{\lambda'}} = \mean{\mode{\herm{b}}{k}{\lambda}\mode{\herm{b}}{k'}{\lambda'}} = 0,
\end{equation}
já que, por exemplo,
\begin{equation}
    \Tr(\mode{b}{k}{\lambda}\rho) = e^{- \beta \omega_k} \Tr(\mode{b}{k}{\lambda}\rho) \implies \Tr(\mode{b}{k}{\lambda}\rho) = 0.
\end{equation}
De forma semelhante, temos
\begin{align}
    \mean{\mode{\herm{b}}{k}{\lambda}\mode{b}{k'}{\lambda'}} &= \Tr(\mode{\herm{b}}{k}{\lambda}\mode{b}{k'}{\lambda'}\rho_\bath)\\
                                                             &= e^{-\beta\omega_{k'}}\Tr(\mode{\herm{b}}{k}{\lambda}\rho_\bath \mode{b}{k'}{\lambda'})\\
                                                             &= e^{-\beta \omega_{k'}} \mean{\mode{b}{k'}{\lambda'}\mode{\herm{b}}{k}{\lambda}}\\
                                                             &= e^{-\beta \omega_{k'}} \left(\delta_{\vetor{k}\vetor{k'}} \delta_{\lambda\lambda'} + \mean{\mode{\herm{b}}{k}{\lambda}\mode{b}{k'}{\lambda'}}\right),
\end{align}
logo
\begin{equation}
    \mean{\mode{\herm{b}}{k}{\lambda}\mode{b}{k'}{\lambda'}} = \delta_{\vetor{k}\vetor{k'}}\delta_{\lambda\lambda'}\frac{1}{e^{\beta \omega_{k}}-1} = \delta_{\vetor{k}\vetor{k'}} \delta_{\lambda\lambda'} N(\omega_k),
\end{equation}
onde \(N(\omega)\) é a distribuição de Planck.

Com estes valores esperados, podemos determinar o tensor de correlação espectral. Primeiro, como \(\mean{\mode{b}{k}{\lambda}} = \mean{\mode{\herm{b}}{k}{\lambda}} = 0,\) temos \(\mean{\vetor{E}(t)} = \vetor{0},\) logo \(\Gamma_{ij}(t, \eta) = \Gamma_{ij}(\eta).\) Segundo, temos
\begin{equation}
    \sumint \dli{s} c_{ij} \delta_{\vetor{k}\vetor{k'}} \delta_{\lambda\lambda'} = \int_0^\infty \dli{s} \sum_{\vetor{k}} \frac{2\pi \omega_k}{L} \deltatr{k}{ij}
\end{equation}
e no limite do contínuo, obtemos
\begin{equation}
    \sumint \dli{s} c_{ij} \delta_{\vetor{k}\vetor{k'}} \delta_{\lambda\lambda'} \to \frac{1}{(2\pi)^2}\int_0^\infty \dli{s} \int_0^\infty \dli{\omega} \omega^3 \int \dli{\Omega} \deltatr{n}{ij}.
\end{equation}
Para avaliar a integral angular, notemos que \(\deltatr{n}{ij}\) é um tensor de segunda ordem sob rotações, portanto o resultado da integral deve ser um tensor de segunda ordem que é invariante por rotações, logo podemos escrever
\begin{equation}
    \int\dli{\Omega} \deltatr{n}{ij} \propto \delta_{ij},
\end{equation}
e então de
\begin{equation}
    \sum_{i = 1}^3 \int \dli{\Omega} \deltatr{n}{ii} = 2 \int \dli{\Omega} = 8\pi,
\end{equation}
obtemos 
\begin{equation}
    \int\dli{\Omega} \deltatr{n}{ij} = \frac{8\pi}{3} \delta_{ij}.
\end{equation}
Desse modo, obtemos
\begin{align}
    \Gamma_{ij}(\eta) &= \frac{2\delta_{ij}}{3\pi} \int_0^\infty \dli{\omega} \omega^3 \int_0^\infty \dli{s} \left\{N(\omega) e^{i(\omega + \eta)s} + \left[N(\omega) + 1\right] e^{-i(\omega - \eta)s}\right\}\\
                      &= \frac{2 \delta_{ij}}{3\pi} \int_0^\infty \dli{\omega} \omega^3 \left\{N(\omega) \int_0^\infty \dli{s} e^{i (\omega + \eta)s} - N(-\omega) \int_0^\infty \dli{s} e^{-i (\omega - \eta)s}\right\}\\
                      &= \frac{2 \delta_{ij}}{3}  \int_0^\infty \dli{\omega} \omega^3 \left\{N(\omega) \left[\delta(\omega + \eta) + \frac{i}{\pi} \mathrm{pv}\frac{1}{\omega + \eta}\right] - N(-\omega)\left[\delta(\omega - \eta) - \frac{i}{\pi}\mathrm{pv}\frac{1}{\omega - \eta}\right]\right\}\\
                      &= \frac{2 \delta_{ij}}{3} \left\{(-\eta)^3N(-\eta)\theta(-\eta) - \eta^3 N(-\eta)\theta(\eta) + \frac{i}{\pi} \mathrm{pv} \int_0^\infty \dli{\omega} \omega^3\left[\frac{N(\omega)}{\omega + \eta} + \frac{N(-\omega)}{\omega - \eta}\right]\right\}\\
                      &= \frac{2\delta_{ij}}{3} \left\{\left[N(\eta) + 1\right]\eta^3 + \frac{i}{\pi} \mathrm{pv} \int_0^\infty \dli{\omega} \omega^3 \left[\frac{N(\omega)}{\omega + \eta} + \frac{N(\omega) + 1}{\eta - \omega}\right]\right\},
\end{align}
onde utilizamos que 
\begin{equation}
    - N(-\omega) = \frac{1}{1 - e^{-\beta \omega}} = \frac{e^{\beta \omega}}{e^{\beta\omega} - 1} = 1 + N(\omega).
\end{equation}
Definimos as quantidades
\begin{equation}
    \gamma(\eta) = \frac{4 \eta^3}{3} \left[N(\eta) + 1\right]\quad\text{e}\quad
    S(\eta) = \frac2{3\pi} \mathrm{pv} \int_0^\infty \dli{\omega} \omega^3 \left[\frac{N(\omega)}{\eta + \omega} + \frac{N(\omega) + 1}{\eta - \omega}\right],
\end{equation}
de forma que
\begin{equation}
    \Gamma_{ij}(\eta) = \delta_{ij} \left[\frac12 \gamma(\eta) + i S(\eta)\right]
\end{equation}
é o tensor de correlação espectral.
    
Como o campo elétrico é linear nos operadores de criação e de aniquilação, temos \(\mean{\vetor{E}(t)} = \vetor{0},\) portanto podemos utilizar as aproximações de Born-Markov. Assim, temos
\begin{align}
    \diff*{\rho(t)}{t} &= \sum_{\omega, \omega'} \sum_{i,j = 1}^3 e^{i(\omega - \omega')t} \delta_{ij} \left[\frac12\gamma(\omega') + i S(\omega')\right] \commutator*{A_i(\omega') \rho(t)}{\herm{A}_j(\omega)} + \mathrm{h.c.}\\
                       &= \sum_{\omega,\omega'} \sum_{j = 1}^3 e^{i(\omega - \omega')t} \left[\frac12 \gamma(\omega') + i S(\omega')\right] \commutator{A_j(\omega') \rho(t)}{\herm{A}_j(\omega)} + \mathrm{h.c.}
\end{align}
logo com a aproximação de onda rotante, obtemos
\begin{equation}
    \diff*{\rho(t)}{t} = -i\commutator{H_\reduced + H_\mathrm{LS}}{\rho(t)} + \mathcal{D} \rho(t),
\end{equation}
onde a correção para o Hamiltoniano é dada por
\begin{equation}
    H_\mathrm{LS} = \sum_{\omega}\sum_{j = 1}^3 S(\omega) \herm{A}_j(\omega) A_j(\omega)
\end{equation}
e o dissipador por
\begin{equation}
    \mathcal{D} \rho(t) = \sum_{\omega} \sum_{j=1}^3 \gamma(\omega) \left[A_j(\omega) \rho(t) \herm{A}_j(\omega) - \frac12\anticommutator*{\herm{A}_j(\omega) A_j(\omega)}{\rho(t)}\right].
\end{equation}

Recordando que \(A_j(-\omega) = \herm{A}_j(\omega)\) e notando que
\begin{equation}
    \gamma(-\omega) = \frac{4(-\omega)^3}{3} \left[N(-\omega) + 1\right] = \frac{4 \omega^3}{3} N(\omega),
\end{equation}
podemos reescrever o dissipador como
\begin{align}
    \mathcal{D} \rho(t) &= \sum_{\omega > 0} \sum_{j = 1}^3 \frac{4 \omega^3}{3} \left[N(\omega) + 1\right]\left[A_j(\omega) \rho(t) \herm{A}_j(\omega) - \frac12\anticommutator*{\herm{A}_j(\omega)A_j(\omega)}{\rho(t)}\right] + \nonumber\\
                        &{}\phantom{=}{}+\sum_{\omega > 0} \sum_{j = 1}^3 \frac{4 \omega^3}{3} N(\omega)\left[\herm{A}_j(\omega) \rho(t) A_j(\omega) - \frac12\anticommutator*{A_j(\omega)\herm{A}_j(\omega)}{\rho(t)}\right],
\end{align}
e vemos que o primeiro tempo corresponde à emissões espontâneas e o segundo à absorções. Como os operadores \(A_j(\omega)\) e \(\herm{A}_j(\omega)\) são auto-operadores do Hamiltoniano livre, concluímos que \(A_j(\omega)\) diminui a energia do sistema reduzido por \(\omega\) e descreve emissões espontâneas e induzidas termicamente com taxa \(\frac{4\omega^3}{3}[N(\omega) + 1]\), enquanto que \(\herm{A}_j(\omega)\) aumenta a energia do sistema reduzido por \(\omega\) e descreve absorções induzidas termicamente com taxa \(\frac{4\omega^3}{3} N(\omega).\)

Para avaliar a validade das aproximações tomadas, consideramos uma interação de dipolo elétrico, com \(\vetor{d}\) sendo o elemento de matriz do operador de dipolo de interesse. A taxa de transição típica \(\gamma_0 = \frac{4\omega^3 \abs{\vetor{d}}^2}{3}\) determina o tempo de relaxamento \(\tau_R \sim \gamma_0^{-1}\) do sistema, enquanto que a frequência típica de transição \(\omega_0\) define o tempo de decaimento \(\tau_B \sim \omega_0^{-1}\) das funções de correlação do ambiente\cite{breuer}. As aproximações de Born-Markov implicam então que \(\gamma_0 \ll \omega_0,\) condizente com a condição de acoplamento fraco.

\section{Decaimento de um sistema de dois níveis}
Consideremos o Hamiltoniano mais simples para o sistema de interesse,
\begin{equation}
    H_\reduced = \frac12 \omega_0 \sigma_3,
\end{equation}
onde 
\begin{equation}
    \sigma_3 = \ketbra{e}{e} - \ketbra{g}{g},
\end{equation}
com \(\ket{e}\) o estado excitado e \(\ket{g}\) o estado fundamental. Os operadores
\begin{equation}
    \sigma_+ = \ketbra{e}{g}\quad\text{e}\quad \sigma_- = \ketbra{g}{e}
\end{equation}
são auto-operadores do Hamiltoniano \(H_\reduced\) com
\begin{equation}
    \commutator{H_\reduced}{\sigma_\pm} = \pm \omega_0 \sigma_\pm.
\end{equation}
Vamos assumir que o operador de dipolo desse sistema admite elementos de matriz não triviais apenas fora da diagonal, por exemplo no caso em que os estados \(\ket{e}\) e \(\ket{g}\) têm paridades distintas. Assim, o operador de dipolo é dado por
\begin{equation}
    \vetor{D}(t) = \vetor{d} e^{-i \omega_0 t} \sigma_- + \vetor{d}^* e^{i \omega_0 t} \sigma_+,
\end{equation}
com \(\vetor{d} = \bra{g} \vetor{D}\ket{e}.\)

Para o sistema de dois níveis, temos
\begin{align}
    H_\mathrm{LS} &= \abs{\vetor{d}}^2 \left[S(\omega_0) \sigma_+ \sigma_- + S(-\omega_0) \sigma_- \sigma_+\right]\\
                  &= \abs{\vetor{d}}^2 \left[S(\omega_0) \ketbra{e}{e} + S(-\omega_0) \ketbra{g}{g}\right]\\
                  &= \abs{\vetor{d}}^2 \left[\bar{S} \unity + \frac{\Delta S}{2} \sigma_3\right],
\end{align}
como a correção do Hamiltoniano, onde
\begin{equation}
    \bar{S} = \frac{S(\omega_0) + S(-\omega_0)}{2}\quad\text{e}\quad
    \Delta S  = S(\omega_0) - S(-\omega_0),
\end{equation}
e
\begin{equation}
    \mathcal{D}\rho(t) = \gamma_0 (N+1) \left[\sigma_- \rho(t) \sigma_+ - \frac12\anticommutator*{\sigma_+\sigma_-}{\rho(t)}\right] + \gamma_0 N \left[\sigma_+\rho(t) \sigma_- - \frac12 \anticommutator*{\sigma_-\sigma_+}{\rho(t)}\right]
\end{equation}
como o dissipador, onde
\begin{equation}
    N = N(\omega_0)\quad\text{e}\quad \gamma_0 = \frac{4 \omega_0^3 \abs{\vetor{d}}^2}{3}.
\end{equation}

O estado do sistema pode ser sempre escrito como
\begin{equation}
    \rho(t) = \frac12 \left[\unity + \vetor{P}(t) \cdot \vetor{\sigma}\right],
\end{equation}
então a contribuição do Hamiltoniano para a equação de movimento é
\begin{align}
    \commutator{H_\reduced + H_\mathrm{LS}}{\rho(t)} &= \left(\frac12 \omega_0 + \frac{\abs{\vetor{d}}^2 \Delta S}{4}\right) P_j(t) \commutator{\sigma_3}{\sigma_j}\\
                                                     &= i\left(\frac12 \omega_0 + \frac{\abs{\vetor{d}}^2 \Delta S}{4}\right) \epsilon_{3jk} P_j(t) \sigma_k\\
                                                     &= i\frac{\alpha}{2}\omega_0 \vetor{e}_3 \cdot [\vetor{P}(t) \times \vetor{\sigma}],
\end{align}
onde
\begin{equation}
    \alpha = 1 + \frac{\abs{\vetor{d}}^2 \Delta S}{2 \omega_0} = 1 + \frac{3 \gamma_0 \Delta S}{8\omega_0^4}.
\end{equation}
Vale notar que no limite de acoplamento fraco temos \(\gamma_0 \ll \omega_0,\) portanto \(\alpha \simeq 1,\) e a separação entre os níveis não é afetada de forma significativa.
Para o dissipador, temos os termos
\begin{equation}
    \sigma_- \rho(t) \sigma_+ = \ketbra{g}{e}\rho(t)\ketbra{e}{g} = \rho_e(t) \sigma_- \sigma_+ = \frac12 [1 + P_3(t)] \sigma_- \sigma_+
\end{equation}
e
\begin{equation}
    \anticommutator{\sigma_\pm \sigma_\mp}{\rho(t)} = \frac14\anticommutator{\unity \pm \sigma_3}{\unity + P_j(t)\sigma_j} = \rho(t) \pm \left(\frac12 \sigma_3 + \frac12 P_3(t) \unity\right),
\end{equation}
portanto
\begin{align}
    \frac1{\gamma_0}\mathcal{D}\rho(t) &= \frac{N+1}{2} \left[\left(1 + P_3(t)\right) \sigma_- \sigma_+ - \frac12 \rho(t) - \frac14 \left(P_3(t)\unity + \sigma_3\right)\right]\nonumber\\
                                       &{}\phantom{=}+\frac{N}{2} \left[\left(1 - P_3(t)\right) \sigma_+ \sigma_- - \frac12 \rho(t) + \frac14 \left(P_3(t)\unity + \sigma_3\right)\right].
\end{align}
Tomando o traço com \(\frac12\sigma_j\) e usando que \(\Tr(\sigma_i \sigma_j) = 2 \delta_{ij},\) obtemos as equações de movimento
\begin{equation}
    \begin{cases}
        \dot{P}_1(t) = - \frac{\alpha}{2} \omega_0 P_2(t) - \frac{\gamma}2 P_1(t)\\
        \dot{P}_2(t) = \frac{\alpha}{2} \omega_0 P_1(t) - \frac{\gamma}2 P_2(t)\\
        \dot{P}_3(t) = - \gamma P_3(t) - \gamma_0,
    \end{cases}
\end{equation}
onde definimos \(\gamma = (2N + 1) \gamma_0\). 

Assim, a solução do sistema de equações diferenciais é dada por
\begin{equation}
    \begin{pmatrix}
        P_1(t)\\
        P_2(t)
    \end{pmatrix}
    = e^{-\frac{\gamma t}{2}}\begin{pmatrix}
        \cos\left(\frac{\alpha}{2} \omega_0 t\right) && -\sin\left(\frac{\alpha}{2} \omega_0 t\right)\\
        \sin\left(\frac{\alpha}{2} \omega_0 t\right) && \cos\left(\frac{\alpha}{2} \omega_0 t\right)
    \end{pmatrix}
    \begin{pmatrix}
        P_1(0)\\
        P_2(0)
    \end{pmatrix}
\end{equation}
e
\begin{equation}
    P_3(t) = \left[\frac{1}{2N+1} + P_3(0)\right] e^{-\gamma t} - \frac{1}{2N+1}.
\end{equation}
A população do estado excitado é dada por
\begin{align}
    \rho_e(t) &= \frac12 + \frac12 P_3(t)\\
              &= \frac12\left[\frac{1}{2N+1} + P_3(0)\right] e^{-\gamma t} + \frac{N}{2N+1}\\
              &= \left[\rho_e(0) - \frac{N}{2N+1}\right]e^{-\gamma t} + \frac{N}{2N+1}
\end{align}
e do estado fundamental por
\begin{equation}
    \rho_g(t) = 1 - \rho_e(t) = \frac{N + 1}{2N + 1} - \left[\frac{N}{2N+1} - \rho_e(0)\right] e^{-\gamma t},
\end{equation}
com as coerências dadas por
\begin{align}
    \rho_{ge}(t) &= \frac{P_1(t) + i P_2(t)}{2}\\
                 &= e^{-\frac{\gamma t}{2}}  \left[\frac{P_1(0) + i P_2(0)}{2}\cos\left(\frac\alpha2 \omega_0 t\right) + \frac{iP_1(0) - P_2(0)}{2} \sin\left(\frac{\alpha}{2} \omega_0 t\right)\right]\\
                 &= e^{-\frac{\gamma t}{2}} \frac{P_1(0) + i P_2(0)}{2} \left[\cos\left(\frac\alpha2 \omega_0 t\right) + i \sin\left(\frac{\alpha}{2} \omega_0 t\right)\right]\\
                 &= e^{-\frac{\gamma t}{2} + i\frac{\alpha}{2} \omega_0 t} \rho_{ge}(0).
\end{align}
No limite em que \(\gamma t \gg 1,\) temos o relaxamento para o estado
\begin{equation*}
    \rho(t) \to \rho_T,
\end{equation*}
onde 
\begin{equation}
    \rho_T = \frac{1}{1 + e^{\beta \omega_0}} \ketbra{e}{e} + \frac{e^{\beta \omega_0}}{1 + e^{\beta \omega_0}}\ketbra{g}{g} = \frac{\exp(-\beta H_\reduced)}{\Tr \exp(- \beta H_\reduced)}
\end{equation}
é o estado de equilíbrio térmico. Por exemplo, para um estado inicial \(\rho(0) = \ketbra{g}{g}\) temos
\begin{equation}
    \rho_e(t) = \frac{N}{2N + 1} \left(1 - e^{- \gamma t}\right)
    \quad\text{e}\quad
    \rho_g(t) = \frac{N + 1}{2N + 1} + \frac{N}{2N + 1} e^{- \gamma t},
\end{equation}
isto é, o estado do sistema reduzido é dado por
\begin{align}
    \rho(t) &= \frac{N}{2N+1}\left(1 - e^{-\gamma t}\right) \ketbra{e}{e} + \left(\frac{N + 1}{2N+1} + \frac{N}{2N + 1}e^{-\gamma t}\right) \ketbra{g}{g}\\
            &= \frac{1 - e^{-\gamma t}}{1 + e^{\beta \omega_0}}\ketbra{e}{e} + \frac{e^{\beta\omega_0} + e^{-\gamma t}}{1 + e^{\beta \omega_0}}\ketbra{g}{g}\\
            &= \rho_T - \frac{e^{-\gamma t}}{1 + e^{\beta\omega_0}}\sigma_3,
\end{align}
com o decaimento exponencial ao estado térmico. Este resultado é um caso particular de que para sistemas ergódicos há o relaxamento para o equilíbrio térmico de um sistema em contato com um banho térmico\cite{breuer}.
